\documentclass{report}[a4paper]
\usepackage[top=35mm,bottom=35mm,left=25mm,right=25mm]{geometry} % Margins

% Decent underlines
\usepackage[normalem]{ulem}

% Hyperreferences
\usepackage{hyperref}

% Imports
\usepackage{import}

% Graphics and images
\usepackage{graphicx} \graphicspath{{./images/}}
\usepackage{subcaption}
\usepackage{float}

% Encodings (to render letters with diacritics and special characters)
\usepackage[utf8]{inputenc}

% Language
\usepackage[english]{babel}

% Source code and algorithms
%\usepackage{amsmath}
\usepackage{algorithm}
\usepackage[noend]{algpseudocode}

% Tables with bold rows
\usepackage{tabularx}
\newcommand\setrow[1]{\gdef\rowmac{#1}#1\ignorespaces}
\newcommand\clearrow{\global\let\rowmac\relax}
\clearrow
\usepackage{multirow}

% Math stuff
\usepackage[mathscr]{euscript}
\usepackage{amssymb, latexsym} %Load math symbols like \blacksquare, but also load normal \leadsto arrows
\usepackage{mathtools} % For \text{...}
% \usepackage{enumitem}
% \usepackage{xcolor}
\newcommand{\expnumber}[2]{{#1}\mathrm{e}{#2}} % scientific notation
\usepackage{siunitx} %SI units

% Definitions, theorems, remarks,...
\usepackage{amsthm}
\newtheorem{definition}{Definition}[section]
\newtheorem{theorem}{Theorem}[section]
\newtheorem{corollary}{Corollary}[theorem]
\newtheorem{lemma}[theorem]{Lemma}
\renewcommand\qedsymbol{$\blacksquare$}
\theoremstyle{remark}
\newtheorem*{remark}{Remark}

% Glossaries
\usepackage[acronym,nonumberlist]{glossaries}
\makeglossaries

\newglossaryentry{heuristic}{
    name=heuristic,
    description={An \emph{heuristic} function $\hat{h}$ of a function $h$ is a function that aims at estimating the value of $h$. In Computer Science, an \emph{heuristic} is a technique to solve a hard problem in a faster way, or to otherwise solve a problem for which there is no known algorithm that can find an exact solution}
}

\newglossaryentry{metaheuristic}{
    name=metaheuristic,
    description={A \emph{metaheuristic} is a strategy to solve certain classes of problems. It is usually a general method to solve a black-box problem, distinguishing it from an \gls{heuristic} which is usually a technique specifically designed to solve a particular instance of a problem}
}

\newglossaryentry{natural selection}{
    name=natural selection,
    description={Key mechanism of biological evolution, consisting of the observation that individuals' characteristics cause differential survival and reproduction of individuals of a species not subjected to intentional environmental stresses}
}

\newglossaryentry{admissible}{
    name=admissible,
    description={An algorithm is \emph{admissible} iff it always returns an optimal solution, if one exists. A function $\hat{h}$ that estimates $h$ is \emph{admissible} iff it always returns a lower bound for $h$ ($\forall x, \hat{h}(x) < h(x)$)}
}

\newglossaryentry{constructive}{
    name=constructive,
    description={An algorithm is \emph{constructive} when it progressively constructs a solution}
}

\newglossaryentry{iterative}{
    name=iterative,
    description={An algorithm is \emph{iterative} when it progressively evaluates and improves a complete solution}
}

\newacronym{DFS}{DFS}{depth-first search}

\newacronym{TSP}{TSP}{travelling salesman problem}

\newacronym{VRP}{VRP}{vehicle routing problem}

\newacronym{SCC}{SCC}{strongly connected component}

\newacronym{TM}{TM}{Turing machine}

% Contents title
\addto\captionsenglish{\renewcommand*\contentsname{Table of contents}}

% Headers and footers
\usepackage{fancyhdr}
\pagestyle{fancyplain}
\fancyhf{}
\lhead{\fancyplain{}{PortoCityTransfers — Delivery I (CAL 2019/20)}}
\rhead{\fancyplain{}{Group T6G05}}
\lfoot{\fancyplain{}{\leftmark}}
\rfoot{\thepage}

% Full reference (number and name of section)
\newcommand*{\fullref}[1]{\hyperref[{#1}]{\ref*{#1} \nameref*{#1}}} % One single link

% Email
\newcommand{\email}[1]{
{\texttt{\href{mailto:#1}{#1}} }
}

% Metadata
\title{\Huge PortoCityTransfers \\ \Large Delivery I \\ \vspace*{4pt} \large CAL 2019/20}
\author{
Group T6G05 \vspace{0.5em} \\
\begin{tabular}{r l}
	\email{up201806429@fe.up.pt} & Diogo Miguel Ferreira Rodrigues        \\
	\email{up201806613@fe.up.pt} & João António Cardoso Vieira e Basto de Sousa \\
	\email{up201806330@fe.up.pt} & Rafael Soares Ribeiro \\
\end{tabular}
}
\date{24th of April, 2020}

% Document
\begin{document}
\maketitle
\setcounter{tocdepth}{2}
\tableofcontents
\listofalgorithms
\clearpage
\printglossary[type=\acronymtype]
\printglossary
\chapter{Introduction} \label{introduction}
This final report details the project designed and implemented in the context of Curricular Unit CAL (Algorithm Design and Analysis).
It fundamentally consists of an application for managing a shuttle service based on a single train station, serving clients in the area inside and surrounding a big city.
\section{Theme description}
The company \emph{PortoCityTransfers} provides shuttle services between Campanhã railway station (Porto, Portugal) and hotels/other places in the Porto region (municipalities of Porto, Matosinhos, Maia, Vila Nova de Gaia). For that end, it owns a certain number of vans, which can transport a given number of passengers each. Service reservation must be made via Internet, where clients can indicate their estimated time of arrival and final destination. Given a set of service requests to satisfy in a given day, and their respective destinations, the company must group passengers and plan routes for that day, having in mind that the passengers should have to wait for a period of time that needs to be as short as possible.\par
This project will design and implement a computer program that will allow \emph{PortoCityTransfers} to plan its routes. The following iterations should be considered for project development:
\begin{enumerate}
    \item The company owns a single van.
    \item The company owns several vans, where multiple travels are simultaneously possible for different vans.
    \item Passengers without reservation must be accounted for, as well as passengers that wish to travel from their homes to the train station.
\end{enumerate}

\chapter{Theoretical notions}
This short chapter presents some basic notions and definitions on Computer Science, which ought to be useful in the following chapters.
\section{Complexity analysis}
Complexity analysis is the process of identifying the computational complexity of an algorithm, describing in a more or less precise way the asymptotic growth of the amount of resources (e.g., time and memory) an algorithm requires.
\begin{definition}[Big-$O$ notation] $f(n) \in O(g(n))$ means that $g$ is an upper bound of $f$.
    \begin{equation*}
        O(g(n))=\{f(n) : \exists c \in \mathbb{R}^+ \colon \exists n_0 \in \mathbb{N} \colon \forall n \geq n_0, f(n) \leq c\cdot g(n)\}
    \end{equation*}
\end{definition}
\begin{definition}[Big-$\Omega$ notation] $f(n) \in \Omega(g(n))$ means that $g$ is a lower bound of $f$.
    \begin{equation*}
        \Omega(g(n))=\{f(n) : \exists c \in \mathbb{R}^+ \colon \exists n_0 \in \mathbb{N} \colon \forall n \geq n_0, f(n) \geq c\cdot g(n)\}
    \end{equation*}
\end{definition}
\begin{definition}[Big-$\Theta$ notation] $f(n) \in \Theta(g(n))$ means that $g$ is a strict upper and lower bound of $f$.
    \begin{equation*}
        \Theta(g(n))=O(g(n)) \cap \Omega(g(n))
    \end{equation*}
\end{definition}
We will not put excessive emphasis on exagerated formalism, so we will by default use the big-$O$ notation.
\section{Graphs}
\subsection{Definitions}
\begin{definition}[Directed weighted graph]
    A directed weighted graph $G$ is a triple $(V, E, w)$, where:
    \begin{itemize}
        \item $V$ is the finite set of \textbf{nodes}.
        \item $E \subseteq V^2$ is the set of \textbf{edges}, where each edge is a pair $(u,v)$ describing an origin and destination.
        \item $w: E \rightarrow \mathbb{R}^+$ is the \textbf{cost function} that maps each edge to a traversal cost.
    \end{itemize}
\end{definition}
\begin{definition}[Transposed graph]
    The transposed graph $G^T(V, E^T)$ of a graph $G(V, E)$ is similar to $G$ but with reversed edge directions:
    \begin{equation*}
        (u, v) \in E \iff (v, u) \in E^T
    \end{equation*}
\end{definition}
\begin{definition}[Path]
    A path $p$ of length $k$ in a graph $G(V,E)$ is a sequence of nodes $\langle p_0,p_1,p_2,...,p_k\rangle$ such that
    \begin{alignat*}{5}
        \text{(All nodes belong to the graph)}       ~~&\forall~0 \leq &&i   &&<    k,&&~p_i \in V \\
        \text{(Node is reachable from previous node)}~~&\forall~0 \leq &&i   &&<    k,&&~(p_i, p_{i+1}) \in E \\
        \text{(No repeated nodes)}                   ~~&\forall~0 \leq &&i,j &&\leq k,&&~i\neq j \implies p_i \neq p_j \\
        \text{(No repeated edges)}                   ~~&\forall~0 \leq &&i,j &&<    k,&&~i\neq j \implies (p_i,p_{i+1}) \neq (p_j, p_{j+1})
    \end{alignat*}
\end{definition}
\begin{definition}[Set of all paths]
    $P(G)$ is the set of all paths in graph $G(V,E)$.
\end{definition}
\begin{definition}[Weight of a path]
    The weight of a path $p$ of length $k$ in graph $G(V,E,w)$ is
    \begin{equation*}
        W(p) = \sum_{i=0}^{k-1}{w(p_i, p_{i+1})}
    \end{equation*}
\end{definition}
\begin{definition}[Adjacency set]
    The adjacency set of $u \in V$ in $G(V,E)$ is the set of nodes directly reachable from $u$:
    \begin{equation*}
        Adj(G, u) = \{(u, v) \in E\} = E \cap (\{v\}\times V)
    \end{equation*}
\end{definition}
\begin{definition}[Descendants] The descendants of a node $s \in V$ in a graph $G(V,E)$ are
    \begin{equation*}
        Desc(s) = Adj(G, s)
    \end{equation*}
\end{definition}
\begin{definition}[Predecessors] The predecessors of a node $s \in V$ in a graph $G(V,E)$ are
    \begin{equation*}
        Pred(s) = Adj(G^T, s)
    \end{equation*}
\end{definition}
\begin{definition}[Strongly connected graph]
    A strongly connected graph is one where there is a path between any two nodes $u, v \in V$.
\end{definition}
\begin{definition}[Strongly connected component]
    A strongly connected component $SCC$ is a subgraph of $G(V, E)$ which is a maximum strongly connected graph (there is no other node $x \not \in SSC$ that can reach and be reached by all nodes $u \in SSC$).
\end{definition}
\begin{definition}[Bridge]
    A bridge in a strongly connected graph $G(V,E)$ is an edge $e \in E$ that, if removed, would disconnect $G$.
\end{definition}
% \section{Miscellaneous}
% \begin{theorem}[Master theorem of Divide-and-Conquer] \label{theor:master}
%     Let $a \geq 1$, $b > 1$ be constants, let $f(n)$ be a function, and let $T(n)$ be defined on the nonnegative integers by the recurrence
%     \begin{equation*}
%         T(n)=a\cdot T(n/b) + f(n)
%     \end{equation*}
%     Then $T(n)$ has the following assymptotic bounds:
%     \begin{enumerate}
%         \item If $f(n) \in O(n^{\log_b a-\epsilon})$ for some constant $\epsilon > 0$, then $T(n) \in \Theta(n^{\log_b a})$
%         \item If $f(n) \in \Theta(n^{\log_b a})$, then $T(n) \in \Theta(f(n) \log n)$
%         \item If $f(n) \in \Omega(n^{\log_b a+\epsilon})$ for some constant $\epsilon > 0$, and $a f(n/b)\leq c f(n)$ for some constant $c<1$ and large $n$, then $T(n) \in \Theta(f(n))$
%     \end{enumerate}
% \end{theorem}
% The proof is prompty available in \cite[p.~97]{intro-alg}.

../report1/problem-formalization.tex
../report1/problem-decomposition.tex
\import{./algorithmic-problems/}{algorithmic-problems.tex}
\chapter{Use cases} \label{use-cases}
Our program should allow a user to:
\begin{itemize}
    \item Visualize the road network, placing nodes in their correct geographical position and displaying speed limits in a graphical way, using the suggested tool \href{https://github.com/STEMS-group/GraphViewer}{GraphViewer} or another visualization tool if necessary.
    \item Visualize the \acrshort{SCC} of the train station.
    \item Find the shortest path from one node to another.
    \item Given a list of vans and services for a day, find the set of routes for that day.
\end{itemize}

\chapter{Conclusion} \label{conclusion}
After briefly presenting the problem statement, we formalized and decomposed the problem into more fundamental problems, which range from simple problems such as reachability, shortest path-finding and \acrshortpl{SCC}, to more complex ones such as the \acrshort{TSP} and \acrshort{VRP}. For each algorithm, we performed complexity analysis and compared results to theoretically evaluate their relative performances and, after considering the pros and cons of each approach, point to a particular solution that will most likely solve our problem in the generally best way.\par
Our fundamental approach is that, once we decompose the main problem into fundamental, known problems, we can then focus on atomic solutions which can be easily tested and benchmarked. We start by solving the shortest path problem, essentially because this is arguably the most important problem in maps: how do I get from a certain place to another in the least time possible?\par
We then go on to build an increasingly complete algorithms database that we will most likely use in the second part of the project, building on atomic solutions to reach successively more complex solutions (for instance, after knowing how to find the shortest path, we need to organize travels so that the system is globally optimized, or at least to a certain extent).\par
During the process of writting this report, we have used numerous concepts familiar to Computer Science and Algorithmics and presented in the context of CAL theoretical classes or researched in the context of this project, such as \emph{\hyperref[glsentry-recursive]{recursion}}, \emph{\hyperref[glsentry-iterative]{iteration}}, \emph{\hyperref[teor:dfs]{mathematical induction}}, \emph{\hyperref[sec:dijkstra-PoC]{cycle variants/invariants}}, \emph{\gls{dynamic programming}}, \emph{\hyperref[glsentry-heuristic]{heuristics}}, \emph{\hyperref[glsentry-greedy]{greedy solutions}}, \emph{\hyperref[algorithm-reachability-dfs]{backtracking}} and \emph{\hyperref[complexity-analysis]{time/memory complexity}}.\par

In the next project delivery we aim at implementing and testing these algorithms, as well as using them in the actual problem we are facing.\par
!REVIEW.
\section{Tasks allocation}
\begin{itemize}
    \item 'D': done
    \item 'R': review
    \item 'D/R': almost done, missing final review
    \item ' ': undone
\end{itemize}
\begin{center}
    \begin{tabular}{l | c | p{29mm} p{30mm} p{29mm}}
        Sections                                        &       & Diogo Rodrigues & João António Sousa & Rafael Ribeiro \\ \hline
        \fullref{introduction}                          & D     & All & -   & -   \\
        \fullref{theoretical-notions}                   & D     & Most & -   & Reachability \\
        \fullref{problem-formalization}                 & D     & All & -   & -   \\
        \fullref{problem-decomposition}                 & D/R   & All & -   & -   \\
        \fullref{algorithm-reachability-dfs}            & D     & All & -   & -   \\
        \fullref{algorithm-shortestpath-floydwarshall}  & D     & -   & All & -   \\
        \fullref{algorithm-shortestpath-dijkstra}       & D     & All & -   & -   \\
        \fullref{algorithm-shortestpath-astar}          & D     & All & -   & -   \\
        \fullref{algorithm-scc-kosaraju}                & D     & Proof of correctness, variant & -   & Most \\
        \fullref{algorithm-scc-tarjan}                  & D     & Proof of correctness & -   & Most \\
        \fullref{algorithm-tsp-heldkarp}                & D     & All & -   & -   \\
        \fullref{algorithm-tsp-nn}                      & D     & -   & All & -   \\
        \fullref{algorithm-vrp-optimal}                 & D     & -   & All & -   \\
        \fullref{algorithm-vrp-heuristic}               & D     & All & -   & -   \\
        \fullref{algorithm-vrp-advanced}                & D     & All & -   & -   \\
        \fullref{use-cases}                             & D/R   & -   & -   & -   \\
        \fullref{conclusion}                            & D/R   & -   & -   & -   \\
    \end{tabular}
\end{center}


\bibliographystyle{acm}
\addcontentsline{toc}{chapter}{Bibliography}
\bibliography{report}
\end{document}
