\documentclass{report}[a4paper]
\usepackage[top=35mm,bottom=35mm,left=25mm,right=25mm]{geometry} % Margins

% Do not put extra space between figures of different chapters in the list of figures
\usepackage{etoolbox}
\makeatletter
\patchcmd{\@chapter}{\addtocontents{lof}{\protect\addvspace{10\p@}}}{}{}{}
\makeatother

% Decent underlines
\usepackage[normalem]{ulem}

% Hyperreferences
\usepackage{hyperref}

% Imports
\usepackage{import}

% Graphics and images
\usepackage{graphicx} \graphicspath{{./images/}}
\usepackage{subcaption}
\usepackage{float}

% Encodings (to render letters with diacritics and special characters)
\usepackage[utf8]{inputenc}

% Language
\usepackage[english]{babel}

% Source code and algorithms
%\usepackage{amsmath}
\usepackage{algorithm}
\usepackage[noend]{algpseudocode}
\usepackage{listings}

% Tables with bold rows
\usepackage{tabularx}
\newcommand\setrow[1]{\gdef\rowmac{#1}#1\ignorespaces}
\newcommand\clearrow{\global\let\rowmac\relax}
\clearrow
\usepackage{multirow}

% Lists and items
\usepackage{enumitem}

% Math stuff
\usepackage[mathscr]{euscript}
\usepackage{amssymb, latexsym} %Load math symbols like \blacksquare, but also load normal \leadsto arrows
\usepackage{mathtools} % For \text{...}
% \usepackage{enumitem}
% \usepackage{xcolor}
\newcommand{\expnumber}[2]{{#1}\mathrm{e}{#2}} % scientific notation
\usepackage{siunitx} %SI units
\newcommand{\degree}{^{\circ}}

% Definitions, theorems, remarks,...
\usepackage{amsthm}
\newtheorem{definition}{Definition}[section]
\newtheorem{theorem}{Theorem}[section]
\newtheorem{corollary}{Corollary}[theorem]
\newtheorem{lemma}[theorem]{Lemma}
\renewcommand\qedsymbol{$\blacksquare$}
\theoremstyle{remark}
\newtheorem*{remark}{Remark}

% Glossaries
\usepackage[acronym,nonumberlist,entrycounter]{glossaries}
\makeglossaries

\newglossaryentry{heuristic}{
    name=heuristic,
    description={An \emph{heuristic} function $\hat{h}$ of a function $h$ is a function that aims at estimating the value of $h$. In Computer Science, an \emph{heuristic} is a technique to solve a hard problem in a faster way, or to otherwise solve a problem for which there is no known algorithm that can find an exact solution}
}

\newglossaryentry{metaheuristic}{
    name=metaheuristic,
    description={A \emph{metaheuristic} is a strategy to solve certain classes of problems. It is usually a general method to solve a black-box problem, distinguishing it from an \gls{heuristic} which is usually a technique specifically designed to solve a particular instance of a problem}
}

\newglossaryentry{natural selection}{
    name=natural selection,
    description={Key mechanism of biological evolution, based on the observation that individuals' characteristics cause differential survival and reproduction of individuals of a species not subjected to intentional environmental stresses}
}

\newglossaryentry{admissible}{
    name=admissible,
    description={An algorithm is \emph{admissible} iff it always returns an optimal solution, if one exists. A function $\hat{h}$ that estimates $h$ is \emph{admissible} iff it always returns a lower bound for $h$ ($\forall x, \hat{h}(x) < h(x)$)}
}

\newglossaryentry{sweep line}{
    name=sweep line,
    description={A \emph{sweep line} algorithm is a generic approach that represents the search space as a two-dimensional space being swiped by an imaginary line. This is usually implemented by first ordering points by their $x$ coordinate and then iterating over them in that order. The most important observation is that, because geometric operations are limited by metrics, all operations needed by the algorithm only require to analyse the problem near the sweep line.}
}

\newglossaryentry{constructive}{
    name=constructive,
    description={An algorithm is \emph{constructive} when it progressively constructs a solution}
}

\newglossaryentry{iterative}{
    name=iterative,
    description={An heuristic algorithm is \emph{iterative} when it progressively evaluates and improves a complete solution. An algorithm is \emph{iterative} when it repeatedly executes a set of instructions without calling itself.}
}

\newglossaryentry{recursive}{
    name=recursive,
    description={An algorithm is \emph{recursive} when it repeatedly executes a set of instructions by calling itself, as opposed to an \gls{iterative} algorithm.}
}

\newglossaryentry{dynamic programming}{
    name=dynamic programming,
    description={Problem-solving strategy generally used to solve algorithmic problems that have optimal substructure (optimal solutions to subproblems lead to globally optimal solutions) and overlapping subproblems (several subproblems are computed from the same sub-subproblems). The distinguishing feature of this approach is memoization (memorization of subproblem solutions for reuse later on. Generally, wherever there is a recurrence, dynamic programming can be applied}
}

\newglossaryentry{greedy}{
    name=greedy,
    description={An algorithm is \emph{greedy} iff it makes a locally optimal choice in the hope that this choice will lead to a globally optimal solution \cite{intro-alg}. Depending on the problem, they may or may not in fact provide a globally optimal solution.}
}

\newglossaryentry{superpolynomial}{
    name=superpolynomial,
    description={A function $f(n)$ is \emph{superpolynomial} iff $\forall k \in \mathbb{R}, f \not \in O(n^k)$ (i.e., it grows faster than polynomial). This concept is particularly useful when $f$ is superpolynomial but not exponential, like $O(2^{\sqrt{n}})$}
}

\newacronym{DFS}{DFS}{depth-first search}

\newacronym{TSP}{TSP}{travelling salesman problem}

\newacronym{VRP}{VRP}{vehicle routing problem}

\newacronym{SCC}{SCC}{strongly connected component}

\newacronym{TM}{TM}{Turing machine}

\newacronym{NN}{NN}{nearest neighbour}

\renewcommand{\glsentrycounterlabel}{}

% Contents title
\addto\captionsenglish{\renewcommand*\contentsname{Table of contents}}

% Headers and footers
\usepackage{fancyhdr}
\pagestyle{fancyplain}
\fancyhf{}
\lhead{\fancyplain{}{PortoCityTransfers — Delivery I (CAL 2019/20)}}
\rhead{\fancyplain{}{Class 6, group 5}}
\lfoot{\fancyplain{}{\leftmark}}
\rfoot{\thepage}

% Full reference (number and name of section)
\newcommand*{\fullref}[1]{\hyperref[{#1}]{\ref*{#1} \nameref*{#1}}} % One single link

% Email
\newcommand{\email}[1]{
{\texttt{\href{mailto:#1}{#1}} }
}

% Metadata
\title{\Huge PortoCityTransfers \\ \Large Delivery I \\ \vspace*{4pt} \large CAL 2019/20}
\author{
Class 6, group 5 \vspace{0.5em} \\
\begin{tabular}{r l}
	\email{up201806429@fe.up.pt} & Diogo Miguel Ferreira Rodrigues        \\
	\email{up201806613@fe.up.pt} & João António Cardoso Vieira e Basto de Sousa \\
	\email{up201806330@fe.up.pt} & Rafael Soares Ribeiro \\
\end{tabular}
}
\date{24th of April, 2020}

% Document
\begin{document}
\maketitle
\setcounter{tocdepth}{2}
\tableofcontents
\listoffigures
\listofalgorithms
\clearpage
\printglossary[type=\acronymtype]
\printglossary
\chapter{Introduction}
This preliminary report details the project to be designed and implemented.
This project fundamentally consists of an application for managing a train station shuttle service based in a single train station, serving clients in the area inside and surrounding a big city.

\chapter{Theoretical notions} \label{theoretical-notions}
This short chapter presents some basic notions and definitions on Computer Science, which ought to be useful in the following chapters.
\section{Complexity analysis} \label{complexity-analysis}
Complexity analysis is the process of identifying the computational complexity of an algorithm, describing in a more or less precise way the asymptotic growth of the amount of resources (e.g., time and memory) an algorithm requires.
\begin{definition}[Big-$O$ notation] $f(n) \in O(g(n))$ means that $g$ is an upper bound of $f$.
    \begin{equation*}
        O(g(n))=\{f(n) \mid \exists c \in \mathbb{R}^+ \colon \exists n_0 \in \mathbb{N} \colon \forall n \geq n_0, f(n) \leq c\cdot g(n)\}
    \end{equation*}
\end{definition}
\begin{definition}[Big-$\Omega$ notation] $f(n) \in \Omega(g(n))$ means that $g$ is a lower bound of $f$.
    \begin{equation*}
        \Omega(g(n))=\{f(n) \mid \exists c \in \mathbb{R}^+ \colon \exists n_0 \in \mathbb{N} \colon \forall n \geq n_0, f(n) \geq c\cdot g(n)\}
    \end{equation*}
\end{definition}
\begin{definition}[Big-$\Theta$ notation] $f(n) \in \Theta(g(n))$ means that $g$ is a strict upper and lower bound of $f$.
    \begin{equation*}
        \Theta(g(n))=O(g(n)) \cap \Omega(g(n))
    \end{equation*}
\end{definition}
We will not put excessive emphasis on exaggerated formalism, so we will by default use the big-$O$ notation.
\section{Graphs} \label{graphs}
\subsection{Definitions}
\begin{definition}[Directed weighted graph]
    A directed weighted graph $G$ is a triple $(V, E, w)$, where:
    \begin{itemize}
        \item $V$ is the finite set of \textbf{nodes}.
        \item $E \subseteq V^2$ is the set of \textbf{edges}, where each edge is a pair $(u,v)$ describing an origin and destination.
        \item $w: E \rightarrow \mathbb{R}^+$ is the \textbf{cost function} that maps each edge to a traversal cost.
    \end{itemize}
\end{definition}
\begin{definition}[Transpose graph]
    The transpose graph $G^T(V, E^T)$ of a graph $G(V, E)$ is similar to $G$ but with reversed edge directions:
    \begin{equation*}
        \forall (u, v) \in E, (v, u) \in E^T
    \end{equation*}
\end{definition}
\begin{definition}[Adjacency set]
    The adjacency set of $u \in V$ in $G(V,E)$ is the set of nodes directly reachable from $u$:
    \begin{equation*}
        Adj(u) = \{(u, v) \in E\} = E \cap (\{v\}\times V)
    \end{equation*}
\end{definition}
\begin{definition}[Path]
    A path $p$ of length $k$ in a graph $G(V,E)$ is a sequence of nodes $\langle p_1,p_2,...,p_{|p|}\rangle$ such that
    \begin{alignat*}{5}
        \text{(All nodes belong to the graph)}       ~~&\forall~1 \leq &&i   &&\leq |p|,&&~p_i \in V \\
        \text{(Consecutive nodes are adjacent)}      ~~&\forall~1 <    &&i   &&\leq |p|,&&~p_i \in Adj(G, p_{i-1})\\
        \text{(No repeated nodes)}                   ~~&\forall~1 \leq &&i,j &&\leq |p|,&&~i\neq j \implies p_i \neq p_j \\
        \text{(No repeated edges)}                   ~~&\forall~1 \leq &&i,j &&<    |p|,&&~i\neq j \implies (p_i,p_{i+1}) \neq (p_j, p_{j+1})
    \end{alignat*}
\end{definition}
\begin{definition}[Subpath]
    A subpath $p'$ of a path $p$ is a subsequence of $p$. From the properties of paths, a subpath is also a path. The subpath of $p$ from indexes $a$ to $b$ ($a \leq b)$ is denoted as $p[a:b]$.
\end{definition}
\begin{definition}[Path concatenation]
    $\langle p^1, p^2 \rangle$ is the concatenation of $p^1$ and $p^2$.
    \begin{equation*}
        \langle p^1, p^2 \rangle = \begin{dcases}
            \langle p^1_1, p^1_2,...,p^1_{|p^1|}, p^2_1,p^2_2,...,p^2_{|p^2|} \rangle & : p_1,p_2 \in P_G \\
            \langle p^1_1, p^1_2,...,p^1_{|p^1|}, p^2                         \rangle & : p_1 \in P_G \wedge p_2 \in V \\
            \langle p^1, p^2_1, p^2_2,..., p^2_{|p^2|}                        \rangle & : p_1 \in V \wedge p_2 \in P_G \\
            \langle p^1, p^2                                                  \rangle & : p_1,p_2 \in V \\
        \end{dcases}
    \end{equation*}
\end{definition}
\begin{definition}[Set of all paths]
    $P_G$ is the set of all paths in graph $G(V,E)$.
\end{definition}
\begin{definition}[Reachability]
    A node $v \in V$ is reachable from $u \in V$ (or \emph{$u$ can reach $v$}) if there is a path starting in $u$ and ending in $v$.
    \begin{equation*}
        u \leadsto v \iff \exists p \in P_G \colon p_1 = u \wedge p_{|p|} = v
    \end{equation*}
    Let us introduce the following notation, which can be read as ``$z$ is reachable from $x$ via $y$":
    \begin{equation*}
        x \leadsto y \wedge y \leadsto z \iff x \leadsto y \leadsto z
    \end{equation*}
\end{definition}
\begin{definition}[Weight of a path]
    The weight of a path $p$ of length $k$ in graph $G(V,E,w)$ is
    \begin{equation*}
        W(p) = \sum_{i=1}^{|p|-1}{w(p_i, p_{i+1})}
    \end{equation*}
\end{definition}
\begin{definition}[Descendants] The descendants $Desc(s)$ of a node $s \in V$ in a graph $G(V,E)$ are all nodes $v \in V$ reachable from $s$.
    \begin{equation*}
        Desc(s) = \{v \in V \mid s \leadsto v\}
    \end{equation*}
\end{definition}
\begin{definition}[Predecessors] The predecessors $Pred(s)$ of a node $s \in V$ in a graph $G(V,E)$ are all nodes $u \in V$ that can reach $s$.
    \begin{equation*}
        Pred(s) = \{u \in V \mid u \leadsto s\}
    \end{equation*}
\end{definition}
\begin{definition}[Strongly connected graph]
    A graph $G(V, E)$ is strongly connected if any node can be reached from any other node.
    \begin{equation*}
        \forall u, v \in V, u \leadsto v
    \end{equation*}
\end{definition}
\begin{definition}[\Acrlong*{SCC}]
    A \acrlong*{SCC} $SCC$ is a subgraph of $G(V, E)$ which is a maximum strongly connected graph (there is no other node $x \not \in SCC$ that can reach and be reached by all nodes $u \in SCC$).
\end{definition}
Let us denote by $SCC(u)$ the \acrshort{SCC} that node $u$ belongs to.
\begin{definition}[Bridge]
    A bridge in a strongly connected graph $G(V,E)$ is an edge $e \in E$ that, if removed, would disconnect $G$.
\end{definition}
\subsection{Theorems}
\begin{theorem}
    The predecessors of $s$ in graph $G(V,E)$ are the descendants of $s$ in the transpose graph $G^T$.
    \begin{equation*}
        Pred_{G}(s)=Desc_{G^T}(s)
    \end{equation*}
\end{theorem}
\begin{proof}
    To prove $Pred_G(s) \subseteq Desc_{G^T}(s)$, we can argue it is trivial that, if $u \leadsto s$ in $G$, then $s \leadsto u$ in $G^T$. Thus, if $u$ is a predecessor of $s$ in $G$ then it is also a descendant of $s$ in $G^T$.\par
    We can follow a similar line of thought to prove $Desc_{G^T}(s) \subseteq Pred_G(s)$.
\end{proof}
\begin{theorem} \label{teor:scc}
    $SCC(p)=Desc(p) \cap Pred(p)$
\end{theorem}
\begin{proof}
    $Desc(p) \cap Pred(p) \subseteq SCC(p)$, given that $\forall u, v \in Desc(p) \cap Pred(p)$ we know $p \leadsto u$, $u \leadsto p$, $p \leadsto v$, $v \leadsto p$, thus meaning $u \leadsto p \leadsto v$ and $v \leadsto p \leadsto u$.\par
    A node $u \in SCC(p)$ must meet both conditions, otherwise it is either false that $p \leadsto u$, or it is false that $u \leadsto p$, thus violating the definition of \acrshort{SCC}.
\end{proof}
\begin{theorem}[Transitivity of reachability]
    The reachability relation is transitive.
    \begin{equation*}
        x \leadsto y \leadsto z \implies x \leadsto z
    \end{equation*}
\end{theorem}
\begin{proof}
Let $p$ be a path that makes $x \leadsto y$ true, and let $q$ be a path that makes $y \leadsto z$ true. The path $r = \langle p[1:|p|], q[2:|q|] \rangle = \langle x, p_2, p_3,...,p_{|p|-1},y,q_2,q_3,...,q_{|q|-1}, z \rangle$ makes $x \leadsto z$ trivially true.
\end{proof}
% \section{Miscellaneous}
% \begin{theorem}[Master theorem of Divide-and-Conquer] \label{theor:master}
%     Let $a \geq 1$, $b > 1$ be constants, let $f(n)$ be a function, and let $T(n)$ be defined on the nonnegative integers by the recurrence
%     \begin{equation*}
%         T(n)=a\cdot T(n/b) + f(n)
%     \end{equation*}
%     Then $T(n)$ has the following assymptotic bounds:
%     \begin{enumerate}
%         \item If $f(n) \in O(n^{\log_b a-\epsilon})$ for some constant $\epsilon > 0$, then $T(n) \in \Theta(n^{\log_b a})$
%         \item If $f(n) \in \Theta(n^{\log_b a})$, then $T(n) \in \Theta(f(n) \log n)$
%         \item If $f(n) \in \Omega(n^{\log_b a+\epsilon})$ for some constant $\epsilon > 0$, and $a f(n/b)\leq c f(n)$ for some constant $c<1$ and large $n$, then $T(n) \in \Theta(f(n))$
%     \end{enumerate}
% \end{theorem}
% The proof is prompty available in \cite[p.~97]{intro-alg}.

\chapter{Problem formalization}
In this chapter, we will make an effort to transform our specific problems into abstract (and to the greatest extent possible, classical) problems, in order to list all subproblems to be solved and possible solutions.
\section{Input}
There are two main inputs to this problem: a description of the map of the area the company serves (section \ref{input-map}); the list of vans (section \ref{input-vans}); the list of requested services for a given day (section \ref{input-services}).
\subsection{Map} \label{input-map}
We will be using maps from OpenStreetMap (OSM) to model the area \emph{PortoCityTransfers} serves.
Maps will be converted into graphs $G=(V,E, w)$, where:
\begin{itemize}
    \item Nodes $V$ represent positions on Earth.
    \item Edges $E$ represent roads traversable by automobiles.
    \item Weight function $w$ is the time it takes to cross an edge. Considering an estimated average distance between successive nodes of about $5 m$, and that traveling at a maximum speed of $120 km/h$ a vehicle would take $0.15 s$ to cross it, cost will be expressed in integer milliseconds which is deemed a unit with more than enough precision, while still being relatively easy to manage as well as fitting in a 32-bit unsigned integer.
\end{itemize}
We will have two graphs:
\begin{itemize}
    \item $G=(V,E,w)$ is the main map, with all the nodes ($|V|=185360$) and edges ($|E|=343617$).
    \item $G'=(V',E',w')$ is the helper map, which instead of having all edges, will only contain as edges what OSM calls "ways", thus allowing for faster path-finding. Thus, the number of nodes ($|V'|=43592$) and edges ($|E'|=54677$) will be significantly smaller.
\end{itemize}
We thus have as input data the directed weighted graph $G=(V,E,w)$.
\subsection{Vans} \label{input-vans}
A list of vans $v_i=(n)$ where $n$ is the capacity.
\subsection{Services} \label{input-services}
For each client, we need to know the origin, destination and the time the person can be picked up.
We thus have as input data a sequence of clients $c$ where each client $c=(u, v, t)$ can be picked at $u$ starting in time $t$, and must be dropped at $v$.
\section{Output}
Various rides $r$, where a ride is a triple $r=(v, C, \langle e \rangle)$. $v$ is the van that will fulfill that ride, $C$ is the set of clients satisfied by that ride and $\langle e \rangle = \langle e_1, e_2,...,e_{|e|} \rangle$ is the sequence of events in that ride. An event $e$ is a triple $e=(t, a, c)$ where $t$ is the time point of that event, $a$ is the type of event and $c$ is defined according to $a$:
\begin{itemize}
    \item $a=+1$: client $c$ will be picked up at $c.u$ at time $t$.
    \item $a=-1$: client $c$ will be dropped off at $c.v$ at time $t$.
    \item $a=0$: the van must be at node $c$ at time $t$.
\end{itemize}
Let $node(e)$ be the node where the van is supposed to be according to event $e$:
\begin{equation*}
    node(e)=\begin{dcases}
        e.c.u & : e.a = +1\\
        e.c.v & : e.a = -1\\
        e.c   & : e.a = 0
    \end{dcases}
\end{equation*}
\section{Restrictions}
The limits of index variables are implicitly limited sequences' sizes.
\begin{enumerate}
    \item $\{r.C : r \text{ is a ride}\}$ is a partition of the set of all clients (a client can only be satisfied once, and all clients must be satisfied).
    \item For a ride $r(v,C,\langle e \rangle)$:
    \begin{enumerate}
        \item It must start and end at the train station: $(e_1.a, e_1.c) = (e_{|e|}.a,e_{|e|}.c)=(0, \text{ train station})$
        \item The clients in $C$ are the same as those in the ride's events: $C=\{e.c : e.a \neq 0\}$
        \item Each client may only be picked up and dropped off once, and all clients must be served:
        \begin{alignat*}{2}
            \forall c \in C,
            &(\exists !e: e.c = c \wedge e.a = +1)\,\wedge \\
            &(\exists !e: e.c = c \wedge e.a = -1)
        \end{alignat*}
        \item The ride must be \emph{feasible}, i.e., there must be enough time between consecutive events for the van to travel from one to the other: $\forall i,\,dist(node(e_i), node(e_{i+1})) \leq e_{i+1}.t - e_i.t$
        \item At every moment, the number of clients inside a van must not be larger than its capacity:
        \begin{equation*}
            \forall i,~\sum_{j=1}^{i}{e_i.a} \leq v.n
        \end{equation*}
    \end{enumerate}
    \item For a van $v$:
    \begin{enumerate}
        \item There are no overlapping rides: $\forall r, r',~r \neq r' \wedge v = v' \iff \min\{e_{|e|}.t,e'_{|e'|}.t\} \leq \max\{e_1.t,e'_1.t\}$
    \end{enumerate}
\end{enumerate}
\section{Objective function}
Let $pick(c)$ be the event where $c$ is picked up, and $drop(c)$ the event where $c$ is dropped off. Our objective function is
\begin{alignat*}{2}
    \text{Total time it took}        ~~& T        &&= \sum_{c \text{ client}}{drop(c).t - pick(c).t} \\
    \text{Minimum time it could take}~~& T_{\min} &&= \sum_{c \text{ client}}{dist(node(drop(c)), node(pick(c)))} \\
    \text{Extra time we must minimize}~~& \Delta T &&= T - T_{\min}
\end{alignat*}
\begin{equation*}
    \min \Delta T
\end{equation*}


\chapter{Problem decomposition} \label{problem-decomposition}
After having formally presented the general problem, we will decompose it into successively smaller, well-known (and preferably classical) problems for each iteration, and present solutions personalized for each iteration.

\section{Iterations} \label{problem-decomposition-iterations}
\subsection{Iteration I}
The company owns one van, and thus can not leave the train station for too long. However, it must also serve the clients that are ready to travel and waiting for van departure.\par
The first iteration will mostly serve the purpose of testing our algorithms, therefore reducing the need for finding near-optimal heuristics for the hardest problems that will eventually be made harder in subsequent iterations.\par
We will use the path-finding algorithms described in section \ref{problem-decomposition-pathfinding} to find the shortest paths between nodes in the graph.\par
One minor heuristic we can implement to group clients is described in section \ref{algorithm-vrp-heuristic}. Having grouped clients, we now need to solve a problem similar to the \acrshort{TSP}, further described in section \ref{problem-decomposition-tsp}.
\subsection{Iteration II}
The company owns a fleet of vans, thus allowing multiple simultaneous travels to handle an increase in client demand. The system must optimize passenger grouping in vans, so as to optimize travels.\par
We will find shortest paths and handle the \acrshort{TSP} in ways similar to the previous iteration. To group clients, we will use strategies generally used to solve the \acrshort{VRP}, as described in section \ref{problem-decomposition-vrp}.
\subsection{Iteration III}
The company may also provide shuttle service for clients that might want to travel from their homes to the train station, as well as for clients without reservation.\par
To solve the first problem, we believe adding some conditions to the TSP/VRP solutions is enough, and return an infinite cost when the solution is impossible.\par
As for the second problem, priority will be given to clients with reservation, meaning a client without reservation will be assigned to one of the vans in the train station at the time of his/her arrival (or the next van that arrives to the station, in case all vans are travelling at the time that client arrives). Additionally, that client will only be accepted if it meets the two following conditions: the van is still able to execute its following travels and the client doesn't add more than a certain time (around $\SI{10}{min}$) to the trip that van was going to make. If there is a van not in a route at that time and that can transport that client while being able to still execute the posterior travels, then a new route is created only for that passenger.\par
If the client has to wait for more than a certain time limit (maybe $\SI{1}{h}$), he/she is ultimately rejected.
\section{Path-finding} \label{problem-decomposition-pathfinding}
We will take advantage of the hierarchical difference between $G$ and $G'$ to make path-finding faster.\par
Since $G'$ is not too large, we can pre-calculate distances between all pairs of nodes in $G'$.
Then, all we have to do is find the two nodes on the ends of the origin's street, and the two nodes on the ends of the destination's street, and try out all the four possibilities.
\section{\texorpdfstring{\Acrlong*{TSP}}{Travelling salesman problem}} \label{problem-decomposition-tsp}
This problem is more thoroughly described in section \ref{algorithm-tsp}.
\section{\texorpdfstring{\Acrlong*{VRP}}{Vehicle routing problem}} \label{problem-decomposition-vrp}
Known strategies for the \acrshort{VRP} offer case-specific solutions taking into account the particular problem's constraints. This problem is more thoroughly described in section \ref{algorithm-vrp}.

\import{./algorithmic-problems/}{algorithmic-problems.tex}
\chapter{Use cases} \label{use-cases}
\section{Graphical uses}
All graphical uses make use of \href{https://github.com/STEMS-group/GraphViewer}{GraphViewer}, and are of the form \texttt{./main \textit{cmd} FRACTION FLAGS}, where:
\begin{itemize}
    \item \texttt{FRACTION} is the fraction of roads to draw; more specifically, if \texttt{FRACTION} is $k$, for a given way $w$ the nodes $w_0, w_k, w_{2k},...,w_{|w|}$ are drawn. Thus, the higher the number, the greater the performance is, but also less nodes/edges are drawn.
    \item \texttt{FLAGS} flags which roads are to be drawn (add them to combine):
    \begin{center}
        \begin{tabular}{l | r}
            \textbf{Type of road} & \textbf{Flag} \\ \hline
            Motorway     &    1 \\
            Trunk        &    2 \\
            Primary      &    4 \\
            Secondary    &    8 \\
            Tertiary     &   16 \\
            Road         &   32 \\
            Residential  &   64 \\
            Slow         &  128
        \end{tabular}
    \end{center}
\end{itemize}
\subsection{View}
\texttt{./main view FRACTION FLAGS} draws the road network, colouring roads according to their designated role (highway, residential, ...). Roads are drawn according to the following scheme:
\begin{center}
    \begin{tabular}{l | l}
        \textbf{Type of road} & \textbf{Colour} \\ \hline
        Motorway              & Red          \\
        Trunk                 & Pink         \\
        Primary               & Orange       \\
        Secondary             & Yellow       \\
        Tertiary              & Gray         \\
        Road                  & Gray         \\
        Residential           & Gray         \\
        Slow                  & Gray, dashed
    \end{tabular}
\end{center}
\subsection{Speed}
\texttt{./main speed FRACTION FLAGS} draws the road network, colouring roads according to their maximum allowed speed. Roads are drawn according to the following scheme:
\begin{center}
    \begin{tabular}{r | l}
        \textbf{Speed [$\SI{}{km/h}$] (up to)} & \textbf{Colour} \\ \hline
                                  120 & Red             \\
                                  100 & Orange          \\
                                   80 & Yellow          \\
                                   60 & Green           \\
                                   50 & Black           \\
                                   40 & Gray            
    \end{tabular}
\end{center}
\subsection{Strongly connected components}
\texttt{./main scc FRACTION FLAGS} draws the road network, colouring roads red if they connect two nodes in the train station's \acrshort{SCC}, or gray if at least one of the nodes is not in the train station's \acrshort{SCC}.
\subsection{Shortest path}
\texttt{./main path FRACTION FLAGS SOUR DEST [-v]} draws the road network, colouring the shortest path (actually least-time path) from nodes \texttt{SOUR} to \texttt{DEST} found by different algorithms. Using the option \texttt{-v} ignores all paths other than the optimal path, and colours nodes if they were explored by certain algorithms.\par
The A* versions are distinguished by the maximum speed they assume a car can go. This is just a method to guarantee we have admissible (only A* algorithm $\SI{90}{km/h}$) and non-admissible heuristics.
\begin{center}
    \begin{tabular}{l | l | l}
        \textbf{Algorithm}            & \textbf{Path colour} & \textbf{Visited nodes colour} \\ \hline
        Dijkstra's algorithm          & Black                & Pink                          \\
        A* algorithm, $\SI{90}{km/h}$ & Black                & Red                           \\
        A* algorithm, $\SI{70}{km/h}$ & Magenta              & Magenta                       \\
        A* algorithm, $\SI{50}{km/h}$ & Blue                 & Blue                          \\
        A* algorithm, $\SI{30}{km/h}$ & Cyan                 & Cyan                          
    \end{tabular}
\end{center}
\section{To implement}
\begin{itemize}
    \item Given a list of vans and services for a day, find the set of routes for that day.
\end{itemize}

\chapter{Conclusion} \label{conclusion}
After briefly presenting the problem statement, we formalized and decomposed the problem into more fundamental problems, which range from simple problems such as reachability, shortest path-finding, \acrshort{SCC} and bridges, to more complex ones such as the \acrshort{TSP} and \acrshort{VRP}. For each algorithm, we performed complexity analysis and compared results to theoretically evaluate their relative performances and, after considering the pros and cons of each approach, point to a specific solution that will most likely solve our problem in the generally best way.\par
Our fundamental approach is that, once we decompose the main problem into fundamental, known problems, we can then focus on atomic solutions which can be easily tested and benchmarked. We start by solving the shortest path problem, essentially because this is arguably the most important problem in maps: how do I get from a certain place to another in the least time possible?\par
We then go on to build an increasingly complete algorithms database that we will most likely use in the second part of the project, building on atomic solutions to reach successively more complex solutions (for instance, after knowing how to find the shortest path, we need to organize travels so that the system is globally optimized, or at least optimized to a certain extent).\par
As for the accessory problems of reachability and bridges, we solve them using well-known algorithms for those purposes.\par
During the process of writting this report, we have used numerous concepts familiar to Computer Science and Algorithmics and presented in the context of CAL theoretical classes or researched in the context of this project, such as \emph{recursivity}, \emph{inductive prooving}, \emph{cycle variants/invariants}, \emph{dynamic programming}, \emph{heuristics}, \emph{greedy solutions} and \emph{backtracking}.\par
!REVIEW.
\section{Tasks allocation}
\begin{itemize}
    \item 'D': done
    \item 'R': review
    \item ' ': undone
\end{itemize}
\begin{center}
    \begin{tabular}{l | c | p{29mm} p{30mm} p{29mm}}
        Sections                                        &       & Diogo Rodrigues & João António Sousa & Rafael Ribeiro \\ \hline
        \fullref{introduction}                          & D     & All & -   & -   \\
        \fullref{theoretical-notions}                   & D     & Most & -   & Reachability \\
        \fullref{problem-formalization}                 & D     & All & -   & -   \\
        \fullref{problem-decomposition}                 & R     & All & -   & -   \\
        \fullref{algorithm-reachability-dfs}            & D     & All & -   & -   \\
        \fullref{algorithm-shortestpath-floydwarshall}  &       & -   & All & -   \\
        \fullref{algorithm-shortestpath-dijkstra}       & D     & All & -   & -   \\
        \fullref{algorithm-shortestpath-astar}          & D     & All & -   & -   \\
        \fullref{algorithm-scc-kosaraju}                & D/R   & Proof of correctness, variant & -   & Most \\
        \fullref{algorithm-scc-tarjan}                  &       & -   & -   & All \\
        \fullref{algorithm-tsp-heldkarp}                & D     & All & -   & -   \\
        \fullref{algorithm-tsp-nn}                      &       & -   & All & -   \\
        \fullref{algorithm-vrp-optimal}                 &       & -   & All & -   \\
        \fullref{algorithm-vrp-heuristic}               & D/R   & All & -   & -   \\
        \fullref{algorithm-vrp-advanced}                & D/R   & All & -   & -   \\
        \fullref{use-cases}                             & D/R   & -   & -   & -   \\
        \fullref{conclusion}                            & D/R   & -   & -   & -   \\
    \end{tabular}
\end{center}


\bibliographystyle{acm}
\addcontentsline{toc}{chapter}{Bibliography}
\bibliography{report}
\end{document}
