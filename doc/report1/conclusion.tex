\chapter{Conclusion} \label{conclusion}
After briefly presenting the problem statement, we formalized and decomposed the problem into more fundamental problems, which range from simple problems such as reachability, shortest path-finding, strongly connected components and bridges, to more complex ones such as the travelling salesman and vehicle routing problems. For each algorithm, we performed complexity analysis and compared results to theoretically evaluate their relative performances and, after considering the pros and cons of each approach, point to a specific solution that will most likely solve our problem in the generally best way.\par
Our fundamental approach is that, once we decompose the main problem into fundamental, known problems, we can then focus on atomic solutions which can be easily tested and benchmarked. We start by solving the shortest path problem, essentially because this is arguably the most important problem in maps: how do I get from a certain place to another in the least time possible?\par
We then go on to build an increasingly complete algorithms database that we will most likely use in the second part of the project, building on atomic solutions to reach successively more complex solutions (for instance, after knowing how to find the shortest path, we need to organize travels so that the system is globally optimized, or at least optimized to a certain extent).\par
As for the accessory problems of reachability and bridges, we solve them using well-known algorithms for those purposes.\par
During the process of writting this report, we have used numerous concepts familiar to Computer Science and Algorithmics and presented in the context of CAL theoretical classes or researched in the context of this project, such as \emph{recursivity}, \emph{inductive prooving}, \emph{cycle variants/invariants}, \emph{dynamic programming}, \emph{heuristics}, \emph{greedy solutions} and \emph{backtracking}.\par
!REVIEW
\section{Tasks allocation}
\begin{itemize}
    \item 'D': done
    \item 'R': review
    \item ' ': undone
\end{itemize}
\begin{center}
    \begin{tabular}{l | c | p{29mm} p{30mm} p{29mm}}
        Sections                                    &       & Diogo Rodrigues & João António Sousa & Rafael Ribeiro \\ \hline
        \fullref{introduction}                      & D     & All & -   & -   \\
        \fullref{theoretical-notions}               & D     & All & -   & Small contribution \\
        \fullref{problem-formalization}             & D     & All & -   & -   \\
        \fullref{problem-decomposition}             &       & All & -   & -   \\
        \fullref{algorithm-reachability-dfs}        & D/R   & All & -   & -   \\
        \fullref{algorithm-shortestpath-floydwarshall} &       & -   & All & -   \\
        \fullref{algorithm-shortestpath-dijkstra}   & D/R   & All & -   & -   \\
        \fullref{algorithm-shortestpath-astar}      & D/R   & All & -   & -   \\
        \fullref{algorithm-tsp-heldkarp}            & D/R   & All & -   & -   \\
        \fullref{algorithm-tsp-nn}                  &       & -   & All & -   \\
        \fullref{algorithm-vrp-optimal}             &       & -   & All & -   \\
        \fullref{algorithm-vrp-heuristic}           & R     & All & -   & -   \\
        \fullref{algorithm-vrp-simannealing}        &       & All & -   & -   \\
        \fullref{algorithm-scc-kosaraju}            &       & -   & -   & All \\
        \fullref{algorithm-scc-tarjan}              &       & -   & -   & All \\
        \fullref{algorithm-scc-dcsc}                & D/R   & All & -   & -   \\
        \fullref{use-cases}                         & D     & -   & -   & -   \\
        \fullref{conclusion}                        & D/R   & -   & -   & -   \\
    \end{tabular}
\end{center}

