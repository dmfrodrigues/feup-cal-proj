\chapter{Problem formalization}
In this chapter, we will make an effort to transform our specific problems into abstract (and to the greatest extent possible, classical) problems, in order to list all subproblems to be solved and possible solutions.
\section{Input}
There are two main inputs to this problem: a description of the map of the area the company serves (section \ref{input-map}); the list of requested services for a given day (section \ref{input-services}).
\subsection{Map} \label{input-map}
We will abstract the area \emph{PortoCityTransfers} serves into a graph $G=(V,E, w)$, where:
\begin{itemize}
    \item Nodes $V$ represent positions on Earth.
    \item Edges $E$ represent roads traversable by automobiles.
    \item Weight function $w$ is the time it takes to cross an edge. Considering an estimated average distance between successive nodes of about $5 m$, and that travelling at a maximum speed of $120 km/h$ a vehicle would take $0.15 s$ to cross it, cost will be expressed in integer milliseconds which is deemed a unit with more than enough precision, while still being relatively easy to manage as well as fitting in a 32-bit unsigned integer.
\end{itemize}
We thus have as input data the directed weighted graph $G=(V,E,w)$.
\subsection{Services} \label{input-services}
For each client, we need to know the origin, destination and the time the person can be picked up.
We thus have as input data a sequence of clients $c$ where each client $c=(u, v, t)$ can be picked at $u$ starting in time $t$, and must be dropped at $v$.
\section{Output}
For each van, a sequence $r$ of all routes. A route $r^i$ is a sequence of triples $r^i_j=(c, t, a)$, meaning the van must be at a certain node to perform a certain action (if $a=+1$ the client must be picked at $c.u$; if $a=-1$ the client must be dropped at $c.v$; if $a=0$ the van should do no particular action, except being at $c$ (which is now only a node, not a client)) at time $t$.
\section{Restrictions}
The limits of the index variables are implicitly limited to the size of the sequences.\par
Let
\begin{equation*}
    v(r^i_j) =
    \begin{cases}
        r^i_j.c.u : a = +1\\
        r^i_j.c.v : a = -1\\
        r^i_j.c   : a = 0
    \end{cases}
\end{equation*}
be the node where the van must be at time $t$.
\begin{enumerate}
    \item Each route starts and ends in the train station; $\forall i, v(r^i_0) = v(r^i_{end}) = $ train station.
    \item Each route must be feasible, meaning for each triple $r^i_j$ in a route the van has enough time to travel to the next triple $r^i_{j+1}$:
    \begin{equation*}
        \forall i,j \in \mathbb{N}, dist(v(p^i_j), v(p^i_{j+1})) \leq p^i_{j+1}.t-p^i_j.t
    \end{equation*}
    \item Routes must be feasible among each other for the same van; for each van with a list of routes $r_i$, a van can only perform the following order after the previous order is completed:
    \begin{equation*}
        \forall i, r^i_{k'}.t \leq r^{i+1}_0.t
    \end{equation*}
    \item All clients must be picked up from their origins and delivered to their destinations in the same route:
    \begin{alignat*}{2}
        \forall c \text{ client}, \exists p \text{ route} \colon \exists i, j~[i<j] \colon
        & p_i.c = c \wedge p_i.a = +1\,\wedge \\
        & p_j.c = c \wedge p_j.a = -1
    \end{alignat*}
    \item No client can be picked up or dropped more than once:
    \begin{gather*}
        \forall p^i_j, p^{i'}_{j'} \text{ triples }[(i,j) \neq (i',j')], p^i_j.c \neq p^{i'}_{j'}.c
    \end{gather*}
\end{enumerate}
\section{Objective function}
Let $pick(c)$ be the triple where $c$ is picked up, and $drop(c)$ the triple where $c$ is dropped off.\par
Our objective function is
\begin{alignat*}{2}
    & T        &&= \sum_{c \text{ client}}{drop(c).t - pick(c).t} \\
    & T_{\min} &&= \sum_{c \text{ client}}{drop(c).t - pick(c).t} \\
    & \Delta T &&= T - T_{\min}
\end{alignat*}
\begin{equation*}
    \min \Delta T
\end{equation*}
\section{Problem decomposition}
For each iteration, we will decompose larger problems into successively smaller, well-known (and preferably classical) problems.
\subsection{Iteration I}
On iteration I, the company owns one van REWRITE
