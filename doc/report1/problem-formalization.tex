\chapter{Problem formalization} \label{problem-formalization}
In this chapter, we will express our specific problems as abstract (and to the greatest extent possible, classical) problems, so we can systematize all subproblems to be solved and possible solutions.
\section{Input} \label{problem-formalization-input}
There are three main inputs to this problem: a description of the map of the area the company serves (section \ref{problem-formalization-input-map}); the list of vans (section \ref{problem-formalization-input-vans}); the list of requested services for a given day (section \ref{problem-formalization-input-services}).
\subsection{Map} \label{problem-formalization-input-map}
We will be using map data extracted from OpenStreetMap (OSM) to model the area \emph{PortoCityTransfers} serves.
The map will be converted into a graph $G(V,E, w)$, where:
\begin{itemize}
    \item Nodes $V$ represent positions on Earth.
    \item Edges $E$ represent roads traversable by automobiles.
    \item Weight function $w$ is the time it takes to cross an edge. Considering an estimated average distance between successive nodes of about $\SI{1}{m}$, and that travelling at a maximum speed of $\SI{120}{km/h}$ a vehicle would take $\SI{0.03}{s}$ to cross it, we initially decided to express time in integer milliseconds; however, as we found out the least time between two consecutive nodes was $\SI{0.0026}{s}$ we decided to express time in integer microseconds, which is deemed a unit with more than enough precision, while still being relatively tangible for human analysis, as well as fitting in a 64-bit integer by a large margin.
\end{itemize}
Map data was extracted on the 20th of March, 2020 (00:53:01 UTC) from \url{www.openstreetmap.org} under the Open Database License (ODbL), using \href{https://wiki.openstreetmap.org/wiki/Overpass_API}{Overpass API} \href{https://wiki.openstreetmap.org/wiki/Overpass_API/versions#Overpass_API_v0.7.56}{v0.7.56.1002}. Data was extracted for the area bounded by latitudes $[41.0022 \degree, 41.2866 \degree]$ and longitudes $[-8.7492 \degree, -8.4945 \degree]$, and saved in XML format as \texttt{map/original/map.xml}.\par
Data was then processed using a short program (\texttt{map/map\_process.cpp}), to extract important data from the large XML map data file ($\SI{175.1}{MB}$) and save it in simple formats and smaller files (\texttt{AMP.nodes}, \texttt{AMP.edges} and \texttt{AMP.points}, $\SI{15.4}{MB}$).

\write18{ if [ ! -f UtWtfIR.png ] ; then curl https://i.imgur.com/UtWtfIR.png -o UtWtfIR.png ; fi ; }
\write18{ if [ ! -f wE0zyNw.png ] ; then curl https://i.imgur.com/wE0zyNw.png -o wE0zyNw.png ; fi ; }
\begin{figure}[H]
    \centering
    \begin{subfigure}{.45\textwidth}
        \centering
        \includegraphics[trim={0 2px 0 2px}, clip, scale=0.30]{UtWtfIR}
        \caption{OpenStreetMap}
    \end{subfigure}
    \begin{subfigure}{.54\textwidth}
        \centering
        \includegraphics[trim={0px 25px 80px 25px}, clip, scale=0.295]{wE0zyNw}
        \caption{Rendered by our program}
    \end{subfigure}
    \caption{Area served by \emph{PortoCityTransfers}, as rendered by OpenStreetMap and our program.}
\end{figure}

\subsection{Vans} \label{problem-formalization-input-vans}
A list of vans, where each van $v=(n)$ \footnote{a van is a 1-tuple} has a capacity of $n$ people, driver not included.
\subsection{Services} \label{problem-formalization-input-services}
For each client, we need to know the origin, destination and the time the person can be picked up.
We thus have as input data a sequence of clients where each client $c=(u, v, t)$ can be picked at $u$ starting in time $t$, and must be dropped at $v$.
\section{Output} \label{problem-formalization-output}
Various rides $r$, where a ride is a triple $r=(v, C, \langle e \rangle)$. $v$ is the van that will fulfill that ride, $C$ is the set of clients satisfied by that ride and $\langle e \rangle = \langle e_1, e_2,...,e_{|e|} \rangle$ is the sequence of events in that ride. An event $e$ is a triple $e=(t, a, c, u)$ where $t$ is the time point of that event, $a$ is the type of event and $c$, $u$ are defined according to $a$:
\pagebreak
\begin{itemize}
    \item $a=+1$: client $c$ will be picked up at $c.u$ at time $t$.
    \item $a=-1$: client $c$ will be dropped off at $c.v$ at time $t$.
    \item $a=0$: the van must be at node $u$ at time $t$.
\end{itemize}
Let $node(e)$ be the node where the van is supposed to be according to event $e$:
\begin{equation*}
    node(e)=\begin{dcases}
        e.c.u & : e.a = +1\\
        e.c.u & : e.a = -1\\
        e.u   & : e.a = 0
    \end{dcases}
\end{equation*}
\section{Restrictions} \label{problem-formalization-restrictions}
The limits of index variables are implicitly limited to the sequences' sizes.
\begin{enumerate}
    \item $\{r.C \mid r \text{ is a ride}\}$ is a partition of the set of all clients (a client can only be satisfied once, and all clients must be satisfied).
    \item For a ride $r(v,C,\langle e \rangle)$:
    \begin{enumerate}
        \item It must start and end at the train station: $(e_1.a, e_1.u) = (e_{|e|}.a,e_{|e|}.u)=(0, \text{ train station})$
        \item The clients in $C$ are the same as those in the ride's events: $C=\{e.c \mid e.a \neq 0\}$
        \item Each client may only be picked up and dropped off once, and all clients must be served:
        \begin{alignat*}{2}
            \forall c \in C,
            &(\exists !e: e.c = c \wedge e.a = +1)\,\wedge \\
            &(\exists !e: e.c = c \wedge e.a = -1)
        \end{alignat*}
        \item The ride must be \emph{feasible}, i.e., there must be enough time between consecutive events for the van to travel from one to the other: $\forall i,\,dist(node(e_i), node(e_{i+1})) \leq e_{i+1}.t - e_i.t$
        \item At every moment, the number of clients inside a van must not be larger than its capacity:
        \begin{equation*}
            \forall i,~\sum_{j=1}^{i}{e_i.a} \leq v.n
        \end{equation*}
    \end{enumerate}
    \item For a van $v$:
    \begin{enumerate}
        \item \label{itm:van-restriction} There are no overlapping rides:
        \begin{equation*}
            \forall r(v,C,\langle e \rangle), r'(v',C',\langle e' \rangle),~r \neq r' \wedge v = v' \iff \min\{e_{|e|}.t,e'_{|e'|}.t\} \leq \max\{e_1.t,e'_1.t\}
        \end{equation*}
    \end{enumerate}
\end{enumerate}
\section{Objective function} \label{problem-formalization-objective}
Our objective is to minimize the time each client needlessly needs to wait to arrive to his/her destination.\par
More formally, let $pick(c)$ be the event where $c$ is picked up, and $drop(c)$ the event where $c$ is dropped off. Our objective function is
\begin{alignat*}{2}
    \text{Total time it took}        ~~& T        &&= \sum_{c \text{ client}}{drop(c).t - pick(c).t} \\
    \text{Minimum time it could take}~~& T_{\min} &&= \sum_{c \text{ client}}{dist(node(drop(c)), node(pick(c)))} \\
    \text{Extra time we must minimize}~~& \Delta T &&= T - T_{\min}
\end{alignat*}
\begin{equation*}
    \min \Delta T
\end{equation*}
