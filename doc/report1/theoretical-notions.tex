\chapter{Theoretical notions} \label{theoretical-notions}
This short chapter presents some basic notions and definitions on Computer Science, which ought to be useful in the following chapters.
\section{Complexity analysis} \label{complexity-analysis}
Complexity analysis is the process of identifying the computational complexity of an algorithm, describing in a more or less precise way the asymptotic growth of the amount of resources (e.g., time and memory) an algorithm requires.
\begin{definition}[Big-$O$ notation] $f(n) \in O(g(n))$ means that $g$ is an upper bound of $f$.
    \begin{equation*}
        O(g(n))=\{f(n) \mid \exists c \in \mathbb{R}^+ \colon \exists n_0 \in \mathbb{N} \colon \forall n \geq n_0, f(n) \leq c\cdot g(n)\}
    \end{equation*}
\end{definition}
\section{Classes of problems} \label{classes-problems}
Although we will not go into much detail over the definition of a \acrfull*{TM}, one should know that, for practical reasons, the C++ language is Turing-complete, which in layman's terms means everytime something applies to a Turing machine it also applies to C++.
\begin{definition}[P]
    Set of all decision problems that can be \textbf{solved in time polynomial} in the problem instance's size by a deterministic \acrshort{TM}.
\end{definition}
\begin{definition}[NP]
    Set of all decision problems that, provided a \emph{witness} $w$ to the answer being \emph{yes}, $w$ can be \textbf{validated in time polynomial} in the problem instance's size by a deterministic \acrshort{TM}.
\end{definition}
\begin{definition}[NP-hard]
    A problem $C$ is NP-hard iff \textbf{every problem in NP can be reduced to $C$}. This is usually expressed as $C$ being at least as hard as the hardest problem in NP. In a stricter definition, $C$ is NP-hard iff every problem in NP can be reduced to $C$, and $C$ does not belong to NP.
\end{definition}
\begin{definition}[NP-complete]
    A problem $C$ is NP-complete iff \textbf{it is in NP and every problem in NP is reducible to $C$}. It is the intersection of NP with NP-hard.
\end{definition}

\section{Graphs} \label{graphs}
\subsection{Definitions}
\begin{definition}[Directed weighted graph]
    A directed weighted graph $G$ is a triple $(V, E, w)$, where
    $V$ is the finite set of \textbf{nodes}, $E \subseteq V^2$ is the set of \textbf{edges} (where each edge is a pair $(u,v)$ describing an origin and destination), and $w: E \rightarrow \mathbb{R}^+$ is the \textbf{cost function} that maps each edge to a traversal cost.
\end{definition}
\begin{definition}[Transpose graph]
    The transpose graph $G^T(V, E^T)$ of a graph $G(V, E)$ is similar to $G$ but with reversed edge directions:
    \begin{equation*}
        E^T = \{(v,u) \mid (u,v) \in E\}
    \end{equation*}
\end{definition}
\begin{definition}[Adjacency set]
    The adjacency set of $u \in V$ in $G(V,E)$ is the set of nodes directly reachable from $u$:
    \begin{equation*}
        Adj(u) = \{(u, v) \in E\} = E \cap (\{v\}\times V)
    \end{equation*}
\end{definition}
\begin{definition}[Path]
    A path $p$ of length $k$ in a graph $G(V,E)$ is a sequence of nodes $\langle p_1,p_2,...,p_{|p|}\rangle$ such that
    \begin{alignat*}{5}
        \text{(All nodes belong to the graph)}       ~~&\forall~1 \leq &&i   &&\leq |p|,&&~p_i \in V \\
        \text{(Consecutive nodes are adjacent)}      ~~&\forall~1 <    &&i   &&\leq |p|,&&~p_i \in Adj(G, p_{i-1})\\
        \text{(No repeated nodes)}                   ~~&\forall~1 \leq &&i,j &&\leq |p|,&&~i\neq j \implies p_i \neq p_j \\
        \text{(No repeated edges)}                   ~~&\forall~1 \leq &&i,j &&<    |p|,&&~i\neq j \implies (p_i,p_{i+1}) \neq (p_j, p_{j+1})
    \end{alignat*}
\end{definition}
\begin{definition}[Subpath]
    A subpath $p'$ of a path $p$ is a subsequence of $p$. From the properties of paths, a subpath is also a path. The subpath of $p$ from indexes $a$ to $b$ ($a \leq b)$ is denoted as $p[a:b]$.
\end{definition}
\begin{definition}[Path concatenation]
    $\langle p^1, p^2 \rangle$ is the concatenation of $p^1$ and $p^2$.
    \begin{equation*}
        \langle p^1, p^2 \rangle = \begin{dcases}
            \langle p^1_1, p^1_2,...,p^1_{|p^1|}, p^2_1,p^2_2,...,p^2_{|p^2|} \rangle & : p^1,p^2 \in P_G \\
            \langle p^1_1, p^1_2,...,p^1_{|p^1|}, p^2                         \rangle & : p^1 \in P_G \wedge p^2 \in V \\
            \langle p^1, p^2_1, p^2_2,..., p^2_{|p^2|}                        \rangle & : p^1 \in V \wedge p^2 \in P_G \\
            \langle p^1, p^2                                                  \rangle & : p^1,p^2 \in V \\
        \end{dcases}
    \end{equation*}
\end{definition}
\begin{definition}[Set of all paths]
    $P_G$ is the set of all paths in graph $G(V,E)$.
\end{definition}
\begin{definition}[Reachability]
    A node $v \in V$ is reachable from $u \in V$ (or \emph{$u$ can reach $v$}) if there is a path starting in $u$ and ending in $v$.
    \begin{equation*}
        u \leadsto v \iff \exists p \in P_G \colon p_1 = u \wedge p_{|p|} = v
    \end{equation*}
\end{definition}
Let us introduce the following notation, which can be read as ``$z$ is reachable from $x$ via $y$":
\begin{equation*}
    x \leadsto y \wedge y \leadsto z \iff x \leadsto y \leadsto z
\end{equation*}
\begin{definition}[Weight of a path]
    The weight of a path $p$ of length $k$ in graph $G(V,E,w)$ is
    \begin{equation*}
        W(p) = \sum_{i=1}^{|p|-1}{w(p_i, p_{i+1})}
    \end{equation*}
\end{definition}
\begin{definition}[Descendants] \label{def:desc} The descendants $Desc(s)$ of a node $s \in V$ in a graph $G(V,E)$ are all nodes $v \in V$ reachable from $s$. The descendants of $s$ can also be referred to as the \emph{transitive closure} of $E$ starting in $s$.
    \begin{equation*}
        Desc(s) = \{v \in V \mid s \leadsto v\}
    \end{equation*}
\end{definition}
\begin{definition}[Predecessors] The predecessors $Pred(s)$ of a node $s \in V$ in a graph $G(V,E)$ are all nodes $u \in V$ that can reach $s$.
    \begin{equation*}
        Pred(s) = \{u \in V \mid u \leadsto s\}
    \end{equation*}
\end{definition}
\begin{definition}[Strongly connected graph]
    A graph $G(V, E)$ is strongly connected if any node can be reached from any other node.
    \begin{equation*}
        \forall u, v \in V, u \leadsto v
    \end{equation*}
\end{definition}
\begin{definition}[\Acrlong*{SCC}]
    A \acrlong*{SCC} $SCC$ is a subgraph of $G(V, E)$ which is a maximum strongly connected graph (there is no other node $x \not \in SCC$ that can reach and be reached by all nodes $u \in SCC$).
\end{definition}
Let us denote by $SCC(u)$ the \acrfull{SCC} that node $u$ belongs to.
\begin{definition}[Bridge]
    A bridge in a strongly connected graph $G(V,E)$ is an edge $e \in E$ that, if removed, would disconnect $G$.
\end{definition}
\subsection{Theorems}
\begin{theorem} \label{theor:pred-desc}
    The predecessors of $s$ in graph $G(V,E)$ are the descendants of $s$ in the transpose graph $G^T$.
    \begin{equation*}
        Pred_{G}(s)=Desc_{G^T}(s)
    \end{equation*}
\end{theorem}
\begin{proof}
    To prove $Pred_G(s) \subseteq Desc_{G^T}(s)$, we can argue it is trivial that, if $u \leadsto s$ in $G$, then $s \leadsto u$ in $G^T$. Thus, if $u$ is a predecessor of $s$ in $G$ then it is also a descendant of $s$ in $G^T$.\par
    We can follow a similar line of thought to prove $Desc_{G^T}(s) \subseteq Pred_G(s)$.
\end{proof}
\begin{theorem} \label{teor:scc}
    $SCC(p)=Desc(p) \cap Pred(p)$
\end{theorem}
\begin{proof}
    $Desc(p) \cap Pred(p) \subseteq SCC(p)$, given that $\forall u, v \in Desc(p) \cap Pred(p)$ we know $p \leadsto u$, $u \leadsto p$, $p \leadsto v$, $v \leadsto p$, thus meaning $u \leadsto p \leadsto v$ and $v \leadsto p \leadsto u$.\par
    A node $u \in SCC(p)$ must meet both conditions, otherwise it is either false that $p \leadsto u$, or it is false that $u \leadsto p$, thus violating the definition of \acrshort{SCC}.
\end{proof}
\begin{theorem}[Transitivity of reachability]
    The reachability relation is transitive.
    \begin{equation*}
        x \leadsto y \leadsto z \implies x \leadsto z
    \end{equation*}
\end{theorem}
\begin{proof}
Let $p$ be a path that makes $x \leadsto y$ true, and let $q$ be a path that makes $y \leadsto z$ true. The path $r = \langle p[1:|p|], q[2:|q|] \rangle = \langle x, p_2, p_3,...,p_{|p|-1},y,q_2,q_3,...,q_{|q|-1}, z \rangle$ makes $x \leadsto z$ trivially true.
\end{proof}
