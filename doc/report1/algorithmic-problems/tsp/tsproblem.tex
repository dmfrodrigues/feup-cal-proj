\section{\texorpdfstring{\Acrlong*{TSP}}{Travelling salesman problem}} \label{algorithm-tsp}
\begin{itemize}
    \item \textbf{Summary:} Given a complete asymmetric weighted graph, find the least cost path that crosses all nodes. 
    \item \textbf{Input:} Graph $G(V,w)$ and origin $s \in V$, where $w=w(S, u, v)$ depends on the visited cities $S$, origin $u$ and destination $v$.
    \item \textbf{Output:} Path $p$ in $G$, starting and ending in $s$ and passing through all nodes.
    \item \textbf{Objective function:} $W(p)~\min$.
\end{itemize}
A variant of this problem is the decision version of the \acrshort*{TSP}, which is formulated as ``is there a path crossing all nodes with cost less than $z$?'', which is NP-complete since a witness in the form of the actual path can be checked to have a cost less or equal to $z$ in linear time.
The general \acrshort*{TSP} is a strict NP-hard problem, since given a solution there is no known algorithm that can check if that solution is optimal (that would actually be as hard as finding the solution itself).\par
However, the rational \acrshort*{TSP} (all edges and solutions are rational) is NP-complete, since it can be converted into an integer \acrshort*{TSP} (where all edges and solutions are integers), which can itself be transformed into the decision version of the \acrshort*{TSP} with an extra step searching the least $z$ for which the decision problem returns \emph{yes}, starting at some finite upper bound that can be easily obtained (this extra section could be a linear search or a binary search, for instance).\par
This does not justify by any means that these instances of the \acrshort*{TSP} are particularly easy to solve; in fact, almost all instances of the \acrshort*{TSP} are pretty hard to solve, and there is no known polynomial algorithm to solve any of these problems.
\subsection{Held-Karp algorithm} \label{algorithm-tsp-heldkarp}
The Held-Karp algorithm is a \gls{dynamic programming} approach to problems similar to the \acrlong*{TSP}, proposed in 1962 independently by Richard Bellman \cite{bellman62} and by Michael Held and Richard Karp \cite{held-karp62}. The main idea is that the current state can be uniquely described by the city $v$ we are at, and the set $S$ of cities we have already visited (except for $s$, which will be the last city to be visited), and for each state we know the minimum cost $D(S,v)$ of reaching $v$ after having visited all cities in $S$. The algorithm then relies on a trivial recurrent formula (although the version we present is \gls{iterative}) to express the links between states, where the solution is $D(V, s)$:
\begin{algorithm}[H]
    \caption{Held-Karp algorithm}
    \label{alg:held-karp}
    \begin{algorithmic}[1]
        \For {$S \in \mathscr{P}(V),\,v \in V$} {$D(S,v)=\infty$}
        \EndFor
        \Function{HK}{$S=0$, $v=s$}
            \If {$D(S,v) \neq \infty$} \Return $D(S,v)$
            \EndIf
            \If {$S = \{v\}$} \Return {$D(S,v) \gets w(0, s, v)$} \Comment{$D(\{v\}, v)=w(\emptyset, s, v)$}
            \EndIf
            \For {$u \in S$}
                \State {$D(S,v) \gets \min\{D(S,v), \Call{HK}{S\backslash \{u\},u})$} \Comment{$D(S, v)=\min_{u \in S}\{D(S\backslash \{v\}, u) + w(S\backslash\{v\},u,v)\}$}
            \EndFor
            \State \Return {$D(S,v)$}
        \EndFunction
    \end{algorithmic}
\end{algorithm}
As we will see, this algorithm has exponential complexity so it is only addequate for very small values of $|V|$, so set $S$ is usually implemented as a bitmask subset of $V$.
\subsubsection{Proof of correctness}
We will rather present an argument as to why these recurrencies are valid. A state can either be:
\begin{itemize}
    \item Of the form $(\{v\}, v)$. In this case, the only cost is to go from $s$ to $v$ with set $\emptyset$.
    \item Not of the form $(\{v\}, v)$. In this case, $v$ is not the only visited city so far, meaning this path has already passed through one of the cities $u \in S$. Therefore, we grab all solutions that go to $u \in S$ and evaluate which one yields the least cost.
\end{itemize}
\subsubsection{Complexity analysis}
There are $2^{|V|}$ subsets of $V$, and $|V|$ possibilities for $v$, meaning there are at most $|V| \cdot 2^{|V|}$ states. Some of these states are not valid or unreachable, but they don't ammount to a value that effectively reduces the time and memory complexities.\par
To find the value of a state $(S,v)$, we will iterate over $|S|$ other states. Since $|S|$ is at most $|V|$, we have a total time complexity of $\Theta (|V|^2 \cdot 2^{|V|})$. The memory complexity for this algorithm is $\Theta (|V| \cdot 2^{|V|})$, since we need to store the result for each state.\par
States of the forms $(S, v)\colon v \not \in S$ are not valid; these ammount to approximately half the states, so we can claim we have approximately $|V|\cdot 2^{|V|-1}$ valid states, and we will have to perform around $|V|^2\cdot 2^{|V|-1}$ operations.\par
\subsubsection{Remarks}
We will have to perform approximately $|V|^2\cdot 2^{|V|-1}$ operations. Using the rule-of-thumb that a computer can perform $10^8$ serial operations per second, and assuming as maximum van capacity the legal limit for european driving license category D1 (light buses with a maximum of 16 seats, not including driver)\cite{dir-2006-126-ec}, we can execute this algorithm in about $\SI{0.08}{s}$, which is acceptable if this algorithm is run only a few times.

\subsection{Nearest neighbour algorithm} \label{algorithm-tsp-nn}
The \acrfull*{NN} algorithm is a \gls{greedy} \gls{constructive} heuristic method for solving the travelling salesman problem.
Starting at the origin $s$, at each step we visit the city nearest to the previous one and add it to the solution until all cities are visited ($S = V$), at which point we finally add $s$ as the final node.
Therefore the memory complexity of this algorithm is $O(|V|)$, and the time complexity of $O(|V|^{2})$ \cite{reinelt}.\par
Let $\overline{S}$ be the complement of $S$ where the universe is $V$.
\begin{algorithm}[h]
    \caption{Nearest-neighbour algorithm}
    \label{alg:nearest neighbour}
    \begin{algorithmic}[1]
        \Function{NN}{$G(V, w)$, $s$}
            \For {$0 \leq i \leq |V|$} {$Ans(i) \gets \textsc{NULL}$}
            \EndFor
            \State {$Ans(0) \gets s$}
            \State {$S \gets \emptyset$}
            \For {$0 < i < |V|$}
                \State {$u' \gets u \in \overline{S} \backslash \{s\}$ that minimizes $w(S, Ans(i-1), u)$}
                \State {$Ans(i) \gets u'$}
                \State {$S \gets S \cup \{u'\}$}
            \EndFor
            \State {$Ans(|V|) \gets s$}
            \State \Return {$Ans$}
        \EndFunction
    \end{algorithmic}
\end{algorithm}
\subsubsection{Remarks}
Even though the \acrlong*{NN} algorithm does not always yield the optimal solution it is still a method that in many cases gives decent results, which can be worked on and improved by other methods.
The solution can be further improved by repeating it $V$ times and executing it for every vertex.\par
In the context of this project it is especially good if we take into account that by finding the closest client's destination to its current position it tries to empty out the van as quickly as possible
which should result in a better result when it comes to customer satisfaction due to a higher number of less displeased customers on one hand and a lower number of customers with a higher degree of unsatisfaction on the other.
\\

