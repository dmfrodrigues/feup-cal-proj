\subsection{Nearest neighbour algorithm} \label{algorithm-tsp-nn}
% Comment do Diogo (esclarecer o que é que ele queria dizer com isto) -> Heuristic
The Nearest Neighbour (NN) algorithm corresponds to a simple to understand constructive heuristic method that attempts to solve the travelling salesman problem.\par
Starting by an empty solution (no cities visited, i.e. $S=\emptyset$), it chooses a random city which will be the starting point $s$. At each step builds upon the previous iteration, visiting the nearest city and adding it to the solution until all the cities have been visited ($S = V$), at which point in time all that is left is to return to the start city.\par
\par.\par.\par.\par Referir aqui brevemente a complexidade.
% doubt: does it connect the last visited to the start city or goes the path backwards?

\vspace{-1em}
\begin{center}
    \begin{algorithm}[ht]
        \caption{Nearest Neighbour algorithm}
        \label{alg:nearest neighbour}
        \begin{minipage}[t]{0.49\linewidth}
            (a) Mathematical version
            \begin{algorithmic}[1]
                \State $s \gets RandomNode \in V$
                \Function{NN}{$S=\{s\}$, $v=s$, $Q \gets V\backslash \{s\}$}
                    \While {$|Q| > 0$}
                        \State $D \gets \min\{D(v, u | u \in Q)\}$
                        \State $Q \gets Q \backslash \{v\}$
                        \State $S \gets S \cup \{v\}$ 
                        \State $s \gets u$ 
                    \EndWhile
                    \State \Return {$D$}
                \EndFunction
            \end{algorithmic}
        \end{minipage}
        \begin{minipage}[t]{0.49\linewidth}
            (b) Programmatic version
            
        \end{minipage}
    \end{algorithm}
\end{center}

\subsubsection{Remarks}
Falar aqui daquilo que o Diogo disse

% Some of the "raw" Bibliography used at the moment. It is here so I don't forget afterwards:
% https://link.springer.com/chapter/10.1007/978-3-319-00951-3_11 -> page 112
% (Wiley Series in Discrete Mathematics & Optimization) E. L. Lawler, Jan Karel Lenstra, A. H. G. Rinnooy Kan, D. B. Shmoys - The Traveling Salesman Problem_ A Guided Tour -> page 150 (page 158 in pdf)
% Gerhard Reinelt. The Travelling Salesman. Computational Solutions for TSP Applications, volume 840 of Lecture Notes in Computer Science. Springer-Verlag, Berlin Heidelberg New York, 1994.

\par
!IN PROGRESS
