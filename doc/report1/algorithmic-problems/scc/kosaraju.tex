\subsection{Kosaraju's algorithm} \label{algorithm-scc-kosaraju}
Kosaraju's algorithm was first sugested by Sambasiva Rao Kosaraju in 1978, but only published in 1981 by Micha Sharir \cite{sharir81}, independently. Like many \acrshort{SCC} algorithms, it performs a \acrshort{DFS} on the graph and another one on its transpose, having as main idea the fact a graph and its transpose have exactly the same \acrshortpl*{SCC}.\par
We consider stack $L$ has its top on the left side, and elements \emph{before} $u$ are closer to the top of the stack (or to the left of $u$).
\begin{algorithm}
    \caption{Kosaraju's algorithm}
    \label{alg-kosaraju}
    \begin{minipage}[t]{0.49\textwidth}
        (a) Main algorithm
        \begin{algorithmic}[1]
            \Function{Kosaraju}{$Adj$}
            \For{$u = 1:|Adj|$} {\Call{DFS\_K}{$Adj$, $u$}}
            \EndFor
            \State {$Adj. \Call{Transpose}{ }$}
            \While{$!L.\Call{Empty}{ }$}
                \State{$u = L. \Call{Pop}{ }$}
                \State \Call{Assign}{$Adj$, $u$, $u$}
            \EndWhile
            \State \Return{$SCC$}
            \EndFunction
        \end{algorithmic}
    \end{minipage}
    \begin{minipage}[t]{0.49\textwidth}
        (b) Helper functions
        \begin{algorithmic}[1]
            \State{$V = \emptyset,~L = \Call{Stack}{ }$}
            \Function{DFS\_K}{$Adj$, $u$}
                \If{$u \in V$} {\Return{}}
                \EndIf
                \State {$V = V \cup \{u\}$}
                \For{$v : Adj[u]$} {\Call{DFS\_K}{$Adj$, $v$}}
                \EndFor
                \State{$L.\Call{Push}{u}$}
            \EndFunction

            \State{$SCC[|Adj|] = \text{NULL}$}

            \Function{Assign}{$Adj$, $u$, $root$}
                \If{$SCC[u] \neq \text{NULL}$} {\Return{}}
                \EndIf
                \State{$SCC[u] = root$}
                \For{$v : Adj[u]]$} {\Call{Assign}{$Adj$, $v$, $root$}}
                \EndFor
            \EndFunction
        \end{algorithmic}
    \end{minipage}
\end{algorithm}
\subsubsection{Proof of correctness}
A formal proof is promply available at \cite[p.~619]{intro-alg}, we will thus only present a brief argument as to why Kosaraju's algorithm is correct. This argument might however seem rather confusing.\par
Let $Before(v)$ be the set of nodes that are before $v$ in $L$ after \textsc{DFS\_K}, let $After(v)$ be the set of nodes that are after $v$ in $L$. Let $P = Pred(v)=P$, $D = Desc(v)$, $B = Before(v)$ and $A=After(v)$.\par
Trivially, we know that,
\begin{equation} \label{eq:ABpart}
    \{B, \{v\}, A\}\text{ is a partition of $V$ in a loose sense}
\end{equation}
since $B$, $\{v\}$ and $A$ are mutually exclusive, and their union is $V$.\par
It is also trivial that
\begin{equation} \label{eq:vnotinPD}
    P \cap \{v\} = D\cap\{v\}=\emptyset
\end{equation}

Let $v$ be a node not yet assigned to an \acrshort*{SCC}, meaning it is the root of its own \acrshort*{SCC}. From now on, we will use the notation \sout{$S$} to state that the set $S$ is empty.\par
Since we have added nodes to $L$ in post-order, for every node $v$ we know all nodes $u$ for which $u \leadsto v$ (all nodes in $P$) are either:
\begin{enumerate}
    \item \label{itm:B} Before $v$ ($B \cap P$);
    \begin{enumerate}
        \item \label{itm:Bexcept} However, it is impossible for a node to be before $v$ and also be in $P \cap D$, because if such node $u$ existed before $v$ then $v$ would have already been assigned to the same \acrshort*{SCC} as $u$ (\sout{$B \cap P \cap D$}).
    \end{enumerate} 
    \item \label{itm:A} After $v$, where it must also be true that $v \leadsto u$ for $u$ to be validly put after $v$ ($A \cap P \cap D$ and \sout{$A \cap P \cap \overline{D}$}).
\end{enumerate}
In the following table, we show a kind of set diagram that shows all possible relevant logical relations (we ignored $\overline{P}$), and which ones are possible.
\begin{center}
    \begin{tabular}{r | c | l}
        In $B$ & & In $A$ \\ \hline
        \begin{tabular}{r}
            from \ref{itm:Bexcept}, \sout{$P \cap D$} \\
            from \ref{itm:B}, $P \cap \overline{D}$
        \end{tabular} & $v$ & \begin{tabular}{l}
            $P \cap D$, from \ref{itm:A} (SCC) \\
            \sout{$P \cap \overline{D}$}, from \ref{itm:A}
        \end{tabular}
    \end{tabular}
\end{center}

As per theorem \ref{teor:scc}, the \acrshort*{SCC} of which $v$ is a root can be obtained from $SCC(v)=P \cap D$. We will now endeavour into proving $P \cap D = P \backslash B$:
\begin{alignat*}{2}
    P \backslash B &= P \cap \overline{B}&&\text{ which from \eqref{eq:ABpart} is} \\
                   &= P \cap (\{v\} \cup A) && \\
                   &= (P \cap \{v\}) \cup (P \cap A)&&\text{ which from \eqref{eq:vnotinPD} is}\\
                   &= \emptyset \cup (P \cap A)&&\\
                   &= P \cap A &&\text{ which from the table is }\\
                   &= A \cap P \cap D &&
\end{alignat*}
From the table, $P \cap D = A \cap P \cap D$.\par
This essentially means we can then perform a \acrshort*{DFS} on $G^T$ to find $P$, being sufficient to test if a newly encountered element of $P$ has already been assigned to a \acrshort*{SCC} (in which case it would be in $B$).\par
There is one last detail we must mention: how we maintain property \ref{itm:A} after processing a node, particularly the fact that \sout{$A \cap P \cap \overline{D}$}. After executing a \acrshort*{DFS} on $G^T$ starting in $v$, we might have marked a node $x \in A$ as belonging to $SCC(v)$ although there is an unprocessed node $\alpha \in A$ before $x$; in this case, before we are finished processing $v$ we can \emph{conceptually} make all nodes marked as belonging to $SCC(v)$ ``bubble'' to the left until all nodes in $SCC(v)$ are placed consecutively in $L$, thus allowing us to now consider all processed nodes to be in $Before(\alpha)$ once we start processing $\alpha$. To better explain this concept, consider $L=\langle u, \alpha, x, \beta, y, \gamma, \delta \rangle $ where greek letters are unprocessed nodes and latin letters are processed nodes that we now know belong to $SCC(u)$; we could then transform $L$ into $L'=\langle u, x, y, \alpha, \beta, \gamma, \delta \rangle$. This is still in post-order, since we know the greek letters are not in $SCC(v)$ (so they're not in $P \cap D$), and $\alpha$ can thus be safely moved from $Before(x)$ to $After(x)$ without breaking the post-order property.\par
In practice, we implement these ``bubble'' movements by simply ignoring them if they are already assigned to \acrshortpl*{SCC}, considering \emph{conceptually} that they were moved to $Before(v)$. These approaches are equivalent, because ``bubbling'' the nodes in $Before(v)$ is equivalent to move them to the top of the stack, which we then pop. 

\subsubsection{Complexity analysis}
Analysing each part of the algorithm, both \textsc{DFS\_K} and \textsc{Assign} have time complexity $O(|E|+|V|)$ since they are fundamentally \acrshortpl*{DFS}. The \textsc{Transpose} operation is also executed in time $O(|E|+|V|)$, thus the total time complexity is $O(|E|+|V|)$.\par
Memory-wise, the visited nodes $V$, the stack $L$ and the resulting \acrshortpl*{SCC} are all $O(|V|)$. On another note, the \textsc{Transpose} operation takes $O(|1|)$ memory if executed in-place, or otherwise $O(|E|+|V|)$ if a copy is made.

% DFS_K O(|V|+|E|)
% Transpose O(|V| + |E|)
% Corridas de Assign O(|V|+|E|) 
% DFS_K e Assign sao fundamentalmente DFS's portanto tem O(|V|+ |E|) (ref à secçao dfs)

% V |V| 
% L |V| (eventualmente todos os nodos vao estar na stack)
% SCC |V| 
% Transpose tem memoria O(1) in-place