\subsection{Kosaraju's algorithm} \label{algorithm-scc-kosaraju}
Kosaraju's algorithm was first sugested by Sambasiva Rao Kosaraju in 1978, but only published in 1981 by Micha Sharir \cite{sharir81}, independently. Like many \acrshort*{SCC} algorithms, it performs a \acrshort{DFS} on the graph and another one on its transpose, having as main idea the fact a graph and its transpose have exactly the same \acrshortpl*{SCC}.
\begin{algorithm}[ht]
    \caption{Kosaraju's algorithm}
    \label{alg-kosaraju}
    \begin{minipage}[t]{0.49\textwidth}
        (a) Main algorithm
        \begin{algorithmic}[1]
            \State{$S = \emptyset,~L = \Call{Stack}{ }$}
            \Function{Kosaraju}{$G(V,E)$}
            \For{$u \in V$} {$SCC(u) \gets \text{NULL}$}
            \EndFor
            \For{$u \in G$} {\Call{DFS\_K}{$G$, $u$}}
            \EndFor
            \While{$!L.\Call{Empty}{ }$} \label{alg:kosaraju-start-cycle}
                \State{$u \gets L. \Call{Pop}{ }$}
                \State \Call{Assign}{$G$, $u$, $u$}
            \EndWhile
            \State \Return{$SCC$}
            \EndFunction
        \end{algorithmic}
    \end{minipage}
    \begin{minipage}[t]{0.49\textwidth}
        (b) Helper functions
        \begin{algorithmic}[1]
            \Function{DFS\_K}{$G(V,E)$, $u$}
                \If{$u \in S$} {\Return{}}
                \EndIf
                \State {$S \gets S \cup \{u\}$}
                \For{$v \in \Call{Adj}{G, u}$} {\Call{DFS\_K}{$G$, $v$}}
                \EndFor
                \State{$L.\Call{Push}{u}$}
            \EndFunction

            
            \Function{Assign}{$G(V,E)$, $u$, $root$}
                \If{$SCC(u) \neq \text{NULL}$} {\Return{}}
                \EndIf
                \State{$SCC(u) \gets root$}
                \For{$v \in \Call{Adj}{G^T, u}$} {\Call{Assign}{$G$, $v$, $root$}}
                \EndFor
            \EndFunction
        \end{algorithmic}
    \end{minipage}
\end{algorithm}
\vspace{-1em}
\subsubsection{Proof of correctness}
A formal proof is promply available at \cite[p.~619]{intro-alg}, we will thus only present an argument as brief as possible as to why Kosaraju's algorithm is correct.\par
Consider the state of $L$ as of line \ref{alg:kosaraju-start-cycle}. Let $P = Pred(v)$, $D = Desc(v)$, $B = Before(v)$ and $A=After(v)$.\par
Trivially, we know that
\begin{gather} \label{eq:ABpart}
    \begin{cases}
        B \cup \{v\} \cup A = V \\
        B \cap \{v\} = \{v\} \cap A = B \cap A = \emptyset
    \end{cases} \implies \overline{B} = \{v\} \cup A \\
    \label{eq:vnotinPD}
    P \cap \{v\} = D\cap\{v\}=\emptyset
\end{gather}
Let $v$ be a node not yet assigned to an \acrshort*{SCC}, meaning it is the root of its own \acrshort*{SCC}. From now on, we will use the notation \sout{$S$} to state that the set $S$ is empty.
Since we have added nodes to $L$ in post-order, for every node $v$ we know all nodes $u$ for which $u \leadsto v$ (all nodes in $P$) are either:
\begin{enumerate}[topsep=0.5em]
    \item \label{itm:B} Before $v$ ($B \cap P$);
    \begin{enumerate}[topsep=0em]
        \item \label{itm:Bexcept} However, it is impossible for a node to be before $v$ and also be in $P \cap D$, because if such node $u$ existed before $v$ then $v$ would have already been assigned to the same \acrshort*{SCC} as $u$ (\sout{$B \cap P \cap D$}).
    \end{enumerate} 
    \item \label{itm:A} After $v$;
    \begin{enumerate}[topsep=0em]
        \item \label{itm:Agood} If $SCC(u)=SCC(v)$ their order is arbitrary, since we can consider $u \leadsto v$ or $v \leadsto u$ ($A \cap P \cap D$).
        \item \label{itm:Abad} The above case is the only valid way of having both $u \leadsto v$ and $u \in A$; otherwise, we would have $u \leadsto v$, $u \not \leadsto v$ and $u \in A$, which is obviously a violation of the fact nodes in $L$ are in pre-order, since $u \leadsto v$ and $u \not \leadsto v$ is undisputable evidence that $u$ must be before $v$ in $L$ (\sout{$A \cap P \cap \overline{D}$}).
    \end{enumerate}
\end{enumerate}
In the following table, we show a kind of set diagram that shows all possible relevant logical relations (we ignored $\overline{P}$), and which ones are possible.
\begin{center}
    \begin{tabular}{r | c | l}
        In $B$ & & In $A$ \\ \hline
        \begin{tabular}{r}
            from \ref{itm:Bexcept}, \sout{$P \cap D$} \\
            from \ref{itm:B}, $P \cap \overline{D}$
        \end{tabular} & $v$ & \begin{tabular}{l}
            $P \cap D$, from \ref{itm:Agood} (SCC) \\
            \sout{$P \cap \overline{D}$}, from \ref{itm:Abad}
        \end{tabular}
    \end{tabular}
\end{center}

As per theorem \ref{teor:scc}, the \acrshort*{SCC} of which $v$ is a root can be obtained from $SCC(v)=P \cap D$. We will now endeavour into proving $P \cap D = P \backslash B$:
\begin{alignat*}{2}
    P \backslash B &= P \cap \overline{B}&&\text{ which from \eqref{eq:ABpart} is} \\
                   &= P \cap (\{v\} \cup A) && \\
                   &= (P \cap \{v\}) \cup (P \cap A)&&\text{ which from \eqref{eq:vnotinPD} is}\\
                   &= \emptyset \cup (P \cap A)&&\\
                   &= P \cap A &&\text{ which from the table is }\\
                   &= A \cap P \cap D &&
\end{alignat*}
From the table, $P \cap D = A \cap P \cap D$.\par
This essentially means we can then perform a \acrshort*{DFS} on $G^T$ to find $P$, being sufficient to test if a newly encountered element of $P$ has already been assigned to a \acrshort*{SCC} (in which case it would be in $B$).\par
There is one last detail we must mention: how, once we start processing $v'$, all nodes before $v'$ are processed and all nodes after $v'$ are unprocessed. It is trivial that all nodes before $v'$ in $L$ are processed, since we process $L$ from left to right. As for the nodes after $v'$: after executing a \acrshort*{DFS} on $G^T$ starting in $v$, we might have marked a node $x \in A$ as belonging to $SCC(v)$ although there is an unprocessed node $\alpha \in A$ before $x$; in this case, before we are finished processing $v$ we can \emph{conceptually} make all nodes marked as belonging to $SCC(v)$ ``bubble'' to the left until all nodes in $SCC(v)$ are placed consecutively in $L$, thus allowing us to now consider all processed nodes to be in $Before(\alpha)$ once we start processing $\alpha$. To better explain this concept, consider $L=\langle v, \alpha, x, \beta, y, \gamma, \delta \rangle $ where greek letters are unprocessed nodes and latin letters are processed nodes that we now know belong to $SCC(v)$; we could then transform $L$ into $L'=\langle v, x, y, \alpha, \beta, \gamma, \delta \rangle$. This is still in post-order, since we know the latin letters are not in $SCC(\alpha)$ (otherwise, $\alpha$ would have been already processed as an element of the \acrshort*{SCC} of the latin letters); since $Before(\alpha) \cap Pred(\alpha) \cap Desc(\alpha)$ is the only set we explicitly said was empty for $\alpha$ (and we just excluded the possibility of the latin letters being in $After(\alpha) \cap Pred(\alpha) \cap Desc(\alpha)$), all latin letters after $\alpha$ can ``bubble'' to $Before(\alpha)$.\par
In practice, we implement these ``bubble'' movements by simply ignoring them if they are already assigned to \acrshortpl*{SCC}, considering \emph{conceptually} that they were moved to $Before(v)$. These approaches are equivalent, because ``bubbling'' the nodes in $Before(v)$ is equivalent to moving them to the top of the stack, which we then pop, which as such means bubbling them and popping them is equivalent to leave them be and once they are found in the stack they are popped and ignored. 

\subsubsection{Complexity analysis}
Analysing each part of the algorithm, both \textsc{DFS\_K} and \textsc{Assign} have time complexity $O(|E|+|V|)$ since they are fundamentally \acrshortpl*{DFS}. The \textsc{Transpose} operation is also executed in time $O(|E|+|V|)$, thus the total time complexity is $O(|E|+|V|)$.\par
Memory-wise, the visited nodes $V$, the stack $L$ and the resulting \acrshortpl*{SCC} are all $O(|V|)$. On another note, the \textsc{Transpose} operation takes $O(|1|)$ memory if executed in-place, or otherwise $O(|E|+|V|)$ if a copy is made.

% DFS_K O(|V|+|E|)
% Transpose O(|V| + |E|)
% Corridas de Assign O(|V|+|E|) 
% DFS_K e Assign sao fundamentalmente DFS's portanto tem O(|V|+ |E|) (ref à secçao dfs)

% V |V| 
% L |V| (eventualmente todos os nodos vao estar na stack)
% SCC |V| 
% Transpose tem memoria O(1) in-place