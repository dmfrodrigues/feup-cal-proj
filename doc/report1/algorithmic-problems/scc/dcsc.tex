\subsection[DCSC algorithm]{Divide-and-Conquer Strong Components algorithm} \label{algorithm-scc-dcsc}
The Divide-and-Conquer Strong Components (DCSC) algorithm is a mostly unknown SCC algorithm, proposed in 2000 by Fleischer et al. \cite{fleischer-dcsc}. Its main ideas are the two lemmas:
\begin{lemma} \label{lem:dcsc2}
    Any SCC is a subset of the descendants of $p$, of the predecessors of $p$, or of the remaining.
\end{lemma}
We will not endeavor into proving lemma \ref{lem:dcsc2} since our implementation does not require it, adding to the fact it is proven in \cite{fleischer-dcsc}.\par
The lecture notes go on to describe how these lemmas can be implemented as a divide-and-conquer algorithm to find all SCCs, however we are more interested in a variant designed by ourselves, to obtain the SCC of node $s$ and supported only on lemma \ref{teor:scc}.
\begin{algorithm}[H]
    \caption{Divide-and-Conquer Strong Components (DCSC) algorithm}
    \label{alg-dcsc}
    \begin{minipage}[t]{0.49\linewidth}
        (a) Original version
        \begin{algorithmic}[1]
            \Function{DCSC}{$G(V,E)$}
                \State {$D \gets \Call{DFS}{G, s}$} \Comment {$Descendants$}
                \State {$P \gets \Call{DFS}{G^T, s}$} \Comment {$Predecessors$}
                \State {$R \gets G \backslash (D \cup P)$} \Comment {$Remainder$}
                \State {$SCC \gets D \cap P$}
                \State {$D' \gets \Call{DCSC}{D \backslash SCC}$}
                \State {$P' \gets \Call{DCSC}{P \backslash SCC}$}
                \State {$R' \gets \Call{DCSC}{R}$}
                \State \Return {$\{SCC\} \cup D' \cup P' \cup R'$}
            \EndFunction
        \end{algorithmic}
    \end{minipage}
    \begin{minipage}[t]{0.49\linewidth}
        (b) Mathematical version
        \begin{algorithmic}[1]
            \Function{DCSCv}{$G(V,E)$, $s$}
                \State {$D \gets \Call{DFS}{G, s}$} \Comment {$Descendants$}
                \State {$P \gets \Call{DFS}{G^T, s}$} \Comment {$Predecessors$}
                \State \Return {$D \cap P$}
            \EndFunction
        \end{algorithmic}
        (c) Programmatic version
        \begin{algorithmic}[1]
            \Function{DCSCv}{$Adj$, $s$}
                \State {$D = \Call{DFS}{Adj, s}$} \Comment {$Descendants$}
                \State {$Adj.\Call{Transpose}{ }$}
                \State {$P = \Call{DFS}{Adj, s}$} \Comment {$Predecessors$}
                \For {$u : V$} {$SCC[u] = D[u] \text{ AND } P[u]$}
                \EndFor
                \State \Return {$SCC$}
            \EndFunction
        \end{algorithmic}
    \end{minipage}
\end{algorithm}
\subsubsection{Proof of correctness}
Both algorithms are trivially correct from lemmas \ref{teor:scc} and \ref{lem:dcsc2}.
\subsubsection{Complexity analysis}
Since we will not be using the original version, the complexity analysis for \textsc{DCSC} is promptly available at \cite{fleischer-dcsc}, which yields the time complexity $\Theta((|E|+|V|) \log |V|)$.\par
For our variant \textsc{DCSCv}, we know that \textsc{DFS} has time complexity $O(|E|+|V|)$, and the \textsc{Transpose} operation can be done in $\Theta(|E|+|V|)$, thus the time complexity of \textsc{DCSC} is $\Theta(|E|+|V|)$.
