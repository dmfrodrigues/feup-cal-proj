\subsection{Floyd-Warshall algorithm} \label{algorithm-shortestpath-floydwarshall}
The Floyd-Warshall's algorithm also knows as Floyd's algorithm was published in 1962 is a generalization of Warshall's algorithm 
(which only tested the existence of a path between two vertices \cite[p.~346]{floyd-alg}) and corresponds to a dynamic programming method 
that finds the shortest path between each pair of vertices in a graph (whose edges' weights may be not only positive but also negative).\par
Assuming $dist(i, j)^{(k)}$ as the distance between $i$ and $j$ that resorts to the vertices $v_{1}$, $v_{2}$, ... $v_{k}$ to create that path
and that $W(i,j)$ is the weight function that returns the weight of the edge that connects the two vertices $i$ and $j$,
this algorithm follows the following recurrence \cite[p.~694]{intro-alg}:

\begin{equation}
    dist[i,j]^{(k)}=\begin{cases}
      W(i,j), & \text{if $k=0$}\\
      min(dist[i,j]^{(k-1)}, dist[i,k]^{(k-1)} + dist[k,j]^{(k-1)}), & \text{if $k>=1$}
    \end{cases}
\end{equation}

The base case corresponds to $k=0$ where $dist(i, j)^{(0)}$ is the value of the edge (if it exists) that connects those two vertices.
Otherwise, the length of the shortest path between vertices $i$ and $j$ is either the length of the path without considering the vertex
labeled as $k$ or the sum of the length of two paths: one from $i$ to $k$ and the other from $k$ to $j$.

\begin{algorithm}[ht]
    \caption{Floyd-Warshall algorithm}
    \label{alg:floyd-warshall}
    \begin{minipage}[t]{0.49\linewidth}
        (a) Mathematical version
        \begin{algorithmic}[1]
            \Function{FloydWarshall}{$G(V,E), W$}
                    \State $dist^{(0)} \gets $ new $ dist[V, V] = \infty$
                    \For{$each\ edge\ E(i,j)$}
                        \State $dist[i,j]^{(0)} \gets W(i,j)$
                    \EndFor
                    \For{$each\ vertex\ i$}
                        \State $dist[i,i]^{(0)} \gets 0$
                    \EndFor
                \For {$0 < k < V$}
                    \State $dist^{(k)} \gets $ new $ dist[V, V]$
                    \For {$0 < i < V$}
                        \For {$0 < j < V$}
                            \State $dist[i,j]^{(k)} \gets min( dist[i,j]^{(k-1)},$
                            $dist[i,k]^{(k-1)} + dist[k,j]^{(k-1)} )$
                        \EndFor
                    \EndFor
                \EndFor
                \State \Return{$dist^{(n)}$}
            \EndFunction
        \end{algorithmic}
    \end{minipage}
    \begin{minipage}[t]{0.49\linewidth}
        (b) Programmatic version
        \begin{algorithmic}[1]
            \Function{FloydWarshall}{$G(V,E), W$}
                \For {$i = 1\ to\ V$}
                    \For {$j = 1\ to\ V$}
                        \State $dist[0][i][j] = \inf$
                        \If{$(i,j) \in E$}
                            \State $dist[0][i][j] = W(i,j)$
                        \EndIf
                    \EndFor
                \EndFor
                \For {$k = 1 < V$}
                    \State $dist^{(k)} \gets $ new $ dist[n, n]$
                    \For {$i = 1\ to\ V$}
                        \For {$j = 1\ to\ V$}
                            \State $dist[k,i,j] = min(dist[k-1,i,j],$
                            $dist[k-1,i,k] + dist[k-1,k,j])$
                        \EndFor
                    \EndFor
                \EndFor
                \State \Return{$dist[k]$}
            \EndFunction
        \end{algorithmic}
    \end{minipage}
\end{algorithm}

\subsubsection{Complexity analysis}
By comparing every path of the graph between each pair of vertices, the Floyd-Warshall algorithm detains a time complexity in the order of $\Theta(|V|^3)$ due not only to the three nested $for$ loops but also to the $O(1)$ run time of line 11.\par
When it comes to memory expenditure, by creating k matrices, one for each iteration of the k for loop, it takes up $\Theta(|V|^3)$ which may possibly be reduced to $\Theta(|V|^2)$ if instead of creating a new matrix, the existing one is updated.

\subsubsection{Remarks}
Even though the Floyd-Warshall's algorithm functions correctly with negative weight edges, these features will most likely not be of much use, due to the nature of the problem:
the time it takes to get from one location to another must be a value equal to (in the case of the ``current position" being equal to the ``final position") or greater than zero.