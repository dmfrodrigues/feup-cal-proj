\subsection{A* algorithm} \label{algorithm-shortestpath-astar}
The A* shortest path algorithm is very well known and widely used for fast path-finding. It was created by Peter Hart, Nils Nilsson and Bertram Raphael of Stanford Research Institute \cite{Astar}, as part of the Shakey project, with the aim of building a mobile robot that could plan its own actions.\par
This algorithm is an extension of the Dijkstra algorithm, using the heuristic function $\hat{h}(u)$ to estimate the distance from $u$ to $d$. In fact, for $\hat{h}(u) = 0$ we trivially obtain Dijkstra's algorithm, which confirms that it makes no assumptions about the graph's context, and that the A* algorithm is an extension of it.
\begin{center}
    \begin{algorithm}[H]
        \caption[A* algorithm]{A* algorithm\\Added lines in green, changed expressions in orange, relative to Dijkstra's algorithm (Algorithm \ref{alg-dijkstra})}
        \label{alg-astar}
        \begin{minipage}[t]{0.49\linewidth}
            (a) Mathematical version
            \begin{algorithmic}[1]
                \Function{Astar}{$G(V,E)$, $s$, $\hat{h}$}
                    \State $Q \gets \emptyset$
                    \For {$v \in V$}
                        \State {$dist(v) \gets \infty$}
                        \State {$\textcolor{green}{\bm{hdist(v) \gets \infty}}$}
                        \State {$prev(v) \gets \text{NULL}$}
                        \State $Q \gets Q \cup \{v\}$
                    \EndFor
                    \State $dist(s) \gets 0$
                    \State $\textcolor{green}{\bm{hdist(s) \gets \hat{h}(s)}}$
                    \While {$|Q| > 0$}
                        \State $u \gets \text{element of } Q \text{ with least } \textcolor{orange}{\bm{hdist}}(u)$
                        \State $Q \gets Q \backslash \{u\}$
                        \For {$v \in \Call{Adj}{G, u}$}
                            \State $c' \gets c + w(u, v)$
                            \If {$c' < dist(v)$}
                                \State $dist(v) \gets c'$
                                \State $\textcolor{green}{\bm{hdist(v) \gets c'+\hat{h}(v)}}$
                                \State $prev(v) \gets u$
                            \EndIf
                        \EndFor
                    \EndWhile
                    \State \Return $dist$, $prev$
                \EndFunction
            \end{algorithmic}
        \end{minipage}
        \begin{minipage}[t]{0.49\linewidth}
            (b) Programmatic version
            \begin{algorithmic}[1]
                \Function{Astar}{$Adj$, $s$, $\hat{h}$}
                    \State $Q = \Call{Min\_Priority\_Queue}{ }$
                    \For {$v : V$}
                        \State {$dist[v] = \infty$}
                        \State {$\textcolor{green}{\bm{hdist[v] = \infty}}$}
                        \State {$prev[v] = \text{NULL}$}
                        \State {$Q.\Call{Insert\_With\_Key}{\infty, v}$}
                    \EndFor
                    \State $dist[s] = 0$
                    \State $\textcolor{green}{\bm{hdist[s] = \hat{h}(s)}}$
                    \State $Q.\Call{Decrease\_Key}{s, \textcolor{orange}{\bm{hdist}}[s]}$
                    \While {$Q.\Call{Size}{ } > 0$}
                        \State $u = Q.\Call{Extract\_Min}{ }$
                        \For {$(v, w): Adj[u]$}
                            \State $c' = c + w$
                            \If {$c' < dist[v]$}
                                \State $dist[v] = c'$
                                \State $\textcolor{green}{\bm{hdist[v] = c'+\hat{h}(v)}}$
                                \State $prev[v] = u$
                                \State $Q.\Call{Decrease\_Key}{v, \textcolor{orange}{\bm{hdist}}[v]}$
                            \EndIf
                        \EndFor
                    \EndWhile
                    \State \Return {$dist$, $prev$}
                \EndFunction
            \end{algorithmic}
        \end{minipage}
    \end{algorithm}
\end{center}
\subsubsection{Proof of correctness}
According to \cite{Astar}, if the heuristic $\hat{h}$ always returns a lower bound for the actual distance $h$ from $u$ to $d$ ($\forall u \in V, \hat{h}(v) \leq h(u)$), then the A* algorithm is guaranteed to find an optimal solution if it exists. This proof is fully presented in the original article so we will only present a brief argument.\par
!ARGUMENT\par
This means the A* algorithm is as effective as Dijkstra's path-finding algorithm so far as $\hat{h}$ provides a lower bound for $h$.
\subsubsection{Complexity analysis}
The poorest choice of heuristic that optimally solves the problem $\hat{h}(u)=0$ has the same time and memory complexities as Dijkstra's algorithm. However, in a graph representing a map (sparse, with uniformly-distributed nodes and random edge weights) and using as heuristic the euclidean distance between $u$ and $d$, it has a complexity of $O(\sqrt{|V|} \cdot (\text{T}_{em}+\text{T}_{dk}))$, given the A* algorithm will trace a more or less straight line to the destination, crossing $\sqrt{|V|}$ nodes which (because the graph is sparse) are adjacent to $O(\sqrt{|V|})$ edges. Because all nodes are still in $Q$, its size is still $|V|$ \footnote{We can implement $Q$ so that it only has nodes with keys different from $\infty$, but that would make no difference in terms of complexity, since we would mean there would still be up to $O(\sqrt{N})$ nodes in $Q$, and we would still have $O(\log |Q|)=O(\log \sqrt{|V|})=O(\frac{1}{2}\log |V|)=O(\log |V|)$}.
If either a binary heap or a Fibonacci heap is chosen, it makes no difference for the total time complexity, which is $O(\sqrt{|V|} \log |V|)$ in both cases.\par
Similarly to Dijkstra's algorithm, the heaps are the only large data structures we use so this algorithm's memory complexity is $O(|V|)$.\par
This means the A* algorithm is at least as fast as Dijkstra's algorithm so far as $\hat{h}$ belongs to the same asymptotic class as $O(\text{T}_{dk})$, which is $O(\log |V|)$ for both types of heaps we analyzed.
\subsubsection{Remarks}
In our case, we can easily point as an heuristic $\hat{h}(u)$ the time it would take the van to go from $u$ to $d$ in a straight line at highway speed, given it gives us a lower bound on the time it would take to actually reach $d$.
