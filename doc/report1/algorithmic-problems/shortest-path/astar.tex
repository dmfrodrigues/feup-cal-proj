\subsection{A* algorithm} \label{algorithm-shortestpath-astar}
The A* shortest path algorithm is very well known and widely used for fast path-finding. It was created by Peter Hart, Nils Nilsson and Bertram Raphael of Stanford Research Institute \cite{Astar}, as part of the Shakey project, with the aim of building a mobile robot that could plan its own actions.\par
This algorithm is an extension of Dijkstra's algorithm, using the heuristic function $\hat{h}(u)$ which serves as extra assumptions about the graph's context to estimate the distance from $u$ to $d$. In fact, for $\hat{h}(u) = 0$ we trivially obtain Dijkstra's algorithm, which confirms that it makes no assumptions, and that the A* algorithm is an extension of it.
\begin{algorithm}[H]
    \caption{A* algorithm}
    \label{alg-astar}
    \begin{algorithmic}[1]
        \Function{Dijkstra}{$G(V,E)$, $s$}
            \For {$v \in V$} \Comment{Initializations}
                \State {$dist(v) \gets \infty$}
                \State {$prev(v) \gets \text{NULL}$}
                \State {$prev(v) \gets \text{NULL}$}
            \EndFor
            \State $Q = V$
            \State {$dist(s) \gets 0$}
            \State $hdist(s) \gets \hat{h}(s)$
            \While {$|Q| > 0$} \Comment{Main cycle}
                \State {$u \gets $ vertex of $Q$ with least $hdist(u)$}
                \For {$v \in \Call{Adj}{G, u}$}
                    \State $c' \gets c + w(u, v)$
                    \If {$c' < dist(v)$}
                        \State $dist(v) \gets c'$
                        \State $hdist(v) \gets c'+\hat{h}(v)$
                        \State $prev(v) \gets u$
                    \EndIf
                \EndFor
            \EndWhile
            \State \Return $dist$, $prev$
        \EndFunction
    \end{algorithmic}
\end{algorithm}
\subsubsection{Proof of correctness}
According to \cite{Astar}, if $\hat{h}$ is \gls{admissible} then the A* algorithm is also \gls{admissible}. This proof is fully presented in the original article so we will only present a brief argument.\par
Our argument consists of a minor change to the argument for Dijkstra's algorithm. In the cycle invariant, the only difference is that, instead of removing the node with least $dist(u)$, we remove the node with the least $hdist(u)$ passing through nodes $u' \not \in Q$, and it can't pass through nodes $u' \in Q$ because there is no node with less $hdist(u')$ that would, even optimistically, allow to reduce $dist(u)$ and $hdist(u)$ even further.\par
This means the A* algorithm is as effective as Dijkstra's path-finding algorithm so far as $\hat{h}$ is admissible.
\subsubsection{Complexity analysis}
The poorest choice of heuristic that optimally solves the problem ($\hat{h}(u)=0$) has the same time and memory complexities as Dijkstra's algorithm. However, in a graph representing a map and using as heuristic a metric similar to the euclidean distance between $u$ and $d$, it has a complexity of $O(\sqrt{|V|} \cdot (\text{T}_{em}+\text{T}_{dk}))$, given the A* algorithm will trace a more or less straight line to the destination, crossing $O(\sqrt{|V|})$ nodes which (because the graph is sparse) are adjacent to $O(\sqrt{|V|})$ edges. Because all nodes are still in $Q$, its size is still $|V|$. If either a binary heap or a Fibonacci heap is chosen, it makes no difference for the total time complexity, which is $O(\sqrt{|V|} \log |V|)$ in both cases.\par
Similarly to Dijkstra's algorithm, the heaps are the only large data structures we use so this algorithm's memory complexity is $O(|V|)$.\par
This means the A* algorithm is at least as fast as Dijkstra's algorithm so far as $\hat{h}$ belongs to the same asymptotic class as $O(\text{T}_{dk})$, which is $O(\log |V|)$ for both types of heaps we analyzed.
\subsubsection{Remarks}
In our case, we can easily point as an heuristic $\hat{h}(u)$ the time it would take the van to go from $u$ to $d$ in a straight line at highway speed, since it is a lower bound on the time it would take to actually reach $d$.
