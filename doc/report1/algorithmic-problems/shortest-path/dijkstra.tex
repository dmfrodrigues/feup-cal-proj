\subsection{Dijkstra's algorithm} \label{algorithm-shortestpath-dijkstra}
Dijkstra's algorithm is one of the most famous graph algorithms, and the most popular shortest path algorithm. It was first published in 1959 by Dutch computer scientist Edsger Dijkstra \cite{dijkstra}.\par
This algorithm finds the distance between a node $s$ and every other node $d$ reachable from $s$ (thus solving all shortest path problems from $s$). The fundamental idea is that, at each moment, we hold a set $Q$ of unprocessed nodes, and we know for all nodes $u$ the least distance $dist(u)$ from $s$ to $u$ and previous node $prev(u)$ in the shortest path (temporary if the node is in $Q$, final otherwise). We then pick the node $u \in Q$ with the least distance to $s$, and process it by iterating over all nodes adjacent to $u$, calculating their potential new distance to $s$ via $u$ ($c'=c+w(u,v)$), and if the new distance is smaller we update $dist(v)=c'$ and $prev(v)=u$.\par
\begin{algorithm}[H]
    \caption{Dijkstra's algorithm}
    \label{alg-dijkstra}
    \begin{minipage}[t]{0.49\linewidth}
        (a) Mathematical version
        \begin{algorithmic}[1]
            \Function{Dijkstra}{$G(V,E)$, $s$}
                \State $Q \gets \emptyset$
                \For {$v \in V$}
                    \State {$dist(v) \gets \infty$}
                    \State {$prev(v) \gets \text{NULL}$}
                    \State $Q \gets Q \cup \{v\}$
                \EndFor
                \State {$dist(s) \gets 0$}
                \While {$|Q| > 0$}
                    \State $u \gets \text{element of } Q \text{ with least } dist(u)$
                    \State $Q \gets Q \backslash \{u\}$
                    \For {$v \in \Call{Adj}{G, u}$}
                        \State $c' \gets c + w(u, v)$
                        \If {$c' < dist(v)$}
                            \State $dist(v) \gets c'$
                            \State $prev(v) \gets u$
                        \EndIf
                    \EndFor
                \EndWhile
                \State \Return $dist$, $prev$
            \EndFunction
        \end{algorithmic}
    \end{minipage}
    \begin{minipage}[t]{0.49\linewidth}
        (b) Programmatic version
        \begin{algorithmic}[1]
            \Function{Dijkstra}{$Adj$, $s$}
                \State $Q = \Call{Min\_Priority\_Queue}{ }$
                \For {$v : V$}
                    \State {$dist[v] = \infty$}
                    \State {$prev[v] = \text{NULL}$}
                    \State {$Q.\Call{Insert\_With\_Key}{\infty, v}$}
                \EndFor
                \State $dist[s] = 0$
                \State $Q.\Call{Decrease\_Key}{s, dist[s]}$
                \While {$Q.\Call{Size}{ } > 0$}
                    \State $u = Q.\Call{Extract\_Min}{ }$
                    \For {$(v, w): Adj[u]$}
                        \State $c' = c + w$
                        \If {$c' < dist[v]$}
                            \State $dist[v] = c'$
                            \State $prev[v] = u$
                            \State $Q.\Call{Decrease\_Key}{v, dist[v]}$
                        \EndIf
                    \EndFor
                \EndWhile
                \State \Return {$dist$, $prev$}
            \EndFunction
        \end{algorithmic}
    \end{minipage}
\end{algorithm}
To achieve the least time complexity possible, the graph is usually represented as an adjacency list $Adj$. We also use a minimum priority queue for $Q$, where the key of $u$ is $dist(u)$, which allows fast extraction of element with lowest key and key decrease when processing adjacent nodes.
\subsubsection{Proof of correctness}
The full proof is rather long and is promptly available (for instance, in \cite[p.~659]{intro-alg}), thus we will only present a short argument for correctness based on cycle invariants/variants for the \textbf{while} loop.
\begin{itemize}
    \item \textbf{Pre-conditions:} $G(V, E, w)$ is a graph. $s \in V$. $Q$ contains all nodes. $dist(s)=0$ and $prev(s)=\text{NULL}$
    \item \textbf{Post-conditions:} $dist(u)$ and $prev(u)$ from $s$ to $\forall u \in V$ are final. Distance of unreachable nodes is $\infty$ and previous node in shortest path is NULL. $prev(s)$ is also NULL.\par
    \item \textbf{Cycle invariant:} $\forall u \not \in Q$, $dist(u)$ and $prev(u)$ are final. $\forall u \in Q$, $dist(u)$ is the least distance from $s$ to $u$ going only through nodes not in $Q$.\par
    This holds for the next cycle run because the element $u$ we remove from $Q$ already has the least $dist(u)$ passing through nodes $u' \not \in Q$, and it can't pass through nodes $u' \in Q$ because there is no node with less $dist(u')$ that would allow to reduce $dist(u)$ even further.
    \item \textbf{Cycle variant:} $|Q|$, which, being a size, is obviously integer. It is strictly decreasing, given that one node is removed from $Q$ for every cycle run. It is also always positive, given it starts with value $|V|$ where $V$ has at least node $s$. The cycle breaks when $|Q| = 0$.
\end{itemize}
\subsubsection{Complexity analysis}
The initial calls to \textsc{Insert\_With\_Key} are made in time $O(|V|)$, given all keys are $\infty$ and the heap property is maintained. In the main cycle, all elementary operations are done in $O(1)$ except \textsc{Extract\_Min} and \textsc{Decrease\_Key}. Let their time complexities be $O(\text{T}_{em})$ and $O(\text{T}_{dk})$ respectively. \textsc{Extract\_Min} is run once for each of the $|V|$ nodes we remove from $Q$. For each node $u$ we iterate over $Adj[u]$, thus iterating once over each of the $|E|$ edges, potentially running \textsc{Decrease\_Key} for each edge. This yields the total time complexity $O(|E|\cdot \text{T}_{dk}+|V|\cdot \text{T}_{em})$. \par
Implementing $Q$ as a binary heap, we have $O(\text{T}_{dk})=O(\log |V|)$ and $O(\text{T}_{em})=O(\log |V|)$, thus yielding the total time complexity $O((|E|+|V|)\log |V|)$. This can be further improved with a Fibonacci heap, with $O(\text{T}_{dk})=O(1)$ and $O(\text{T}_{em})=O(\log |V|)$ (both amortized), thus yielding $O(|E| + |V|\log |V|)$.\par
The algorithm takes as much memory as the heaps, thus giving a memory complexity of $O(|V|)$.
