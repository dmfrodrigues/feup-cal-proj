\subsection{Vertical stripes algorithm} \label{algorithm-vstripes}
This algorithm was developed and named by us, as a simple and reliable alternative to searching the web for non-optimal solutions, or solutions so complex their implementation would be impossible in the available time.\par
We start by making an assumption: for all pivots we will evaluate, the closest point is at a distance of at most $\Delta$ from the closest point in $S$. Now, we divide our 2D space into several non-overlapping vertical stripes, each of length $\Delta$, where the first stripe starts in the minimum $x$ coordinate and the last stripe ends in or after the maximum $x$ coordinate, where each stripe maintains the set of points it contains ordered by the $y$ coordinate (e.g., for $S=\{(0,0),(1,0),(2,0),(3,0)\}$ and $\Delta = 1.75$, we would have stripes $[0,1.75)$ and $[1.5, 3.5)$ containing points $\{(0,0),(1,0)\}$ and $\{(2,0),(3,0)\}$ respectively).\par
Now for finding the solution for a given pivot $p$, we find the stripe it belongs to, find its hypothetical position in that stripe and get the previous and next points in the ordered set, which will allow us to make a first estimate of the minimum distance, $d_0$. Now, we test all nodes in that stripe and also the stripes before and after the stripe of $p$, but only those for which $y_p - d_0 \leq y \leq y_p + d_0$.
\begin{algorithm}[h]
    \caption{\Acrlong*{DFS}}
    \label{alg-vstripes}
    \begin{algorithmic}[1]
        \Function{VStripes\_Init}{$S$, $\Delta$}
            \State {$x_{\min} \gets -\infty$}
            \State {$x_{\max} \gets +\infty$}
            \For {$q \in S$}
                \State {$x_{\min} \gets \min{x_{\min}, x_q}$}
                \State {$x_{\max} \gets \max{x_{\max}, x_q}$}
            \EndFor
            \State {$lim \gets \langle x{\min} \rangle$}
            \State {$x \gets x_{\min}$}
            \While {$lim_{|lim|-1} < x_{\max}$} {$lim_{|lim|} \gets (x += \Delta)$}
            \EndWhile
            \State {$stripes \gets \langle \emptyset, \emptyset, ... \rangle$} \Comment{Size $|lim|-1$}
            \For{$q \in S$} {$stripes_{\lfloor (x_q-x_{\min})/\Delta \rfloor} \gets stripes_{\lfloor (x_q-x_{\min})/\Delta \rfloor} \cup \{q\}$}
            \EndFor
        \EndFunction
        \Function{VStripes}{$p$}
            \State {$p' \gets \text{NULL}$}
            \State {$d \gets +\infty$}
            \State {$i \gets \lfloor (x_p-x_{\min})/\Delta \rfloor$}
            \State {$a \gets \text{point before $p$ in $stripes_i$}$}
            \State {$b \gets \text{point after $p$ in $stripes_i$}$}
            \If{$a \neq \text{NULL}~\text{and}~dist(a, p) < d$} {$p' \gets a,~d \gets dist(a,p)$}
            \EndIf
            \If{$b \neq \text{NULL}~\text{and}~dist(b, p) < d$} {$p' \gets b,~d \gets dist(b,p)$}
            \EndIf
            \For {$q \in (stripes_{i-1} \cup stripes_{i} \cup stripes_{i+1}) \cap [x_q-d, x_q+d]$}
                \If{$dist(p,q) < d$} {$p' \gets q,~d \gets dist(p,q)$}
                \EndIf
            \EndFor
            \State \Return $p'$
        \EndFunction
    \end{algorithmic}
\end{algorithm}
\subsubsection{Proof of correctness}
If the pre-condition about $\Delta$ is true, and the pivot is in the minimum rectangle containing all points, then the algorithm is trivially correct, since we do not need to search more than $\Delta$ to each side of $p$.
\subsubsection{Complexity analysis}
Let $\Delta x$ be the $x$ coordinate amplitude. Initialization takes $O(|S|)$ for the first cycle, $O(\Delta x / \Delta)$ for the second cycle and $O(|S| \log |S|)$ for the third, thus having a total time complexity of $O(\Delta x / \Delta + |S| \log |S|)$. As for memory complexity, $lim$ takes $O(\Delta x / \Delta)$ space, and $stripes$ takes $O(\Delta x / \Delta + |S|)$, thus having a total memory complexity of $O(\Delta x / \Delta + |S|)$.\par
When answering a query, we only require $O(1)$ additional memory. We also only require time $O(log |S|)$, if we can be convinced by an argument similar to that used to find the complexity of the \gls{sweep line} algorithm for closest pair of points \footnote{In the closest pair of points problem, one must find two points in a set $S$ that are closest to each other. \href{https://www.cs.mcgill.ca/~cs251/ClosestPair/ClosestPairPS.html}{A possible solution} consists of a sweep line approach where we iterate over all nodes ordered by $x$, always keeping a bounding box of dimensions $d \times 2d$ of points that are potentially closer than $d$ to the point being currently analysed. One of the fundamental parts of the complexity analysis of this problem is the fact the bounding box always contains $O(1)$ nodes, which we will not discuss any further since it falls outside the scope of this project.}.
\subsubsection{Remarks}
Assumptions that all pivots are at a distance of at most $\Delta$ from the closest point in $S$ and that all query pivots are bounded by the points of set $S$ are reasonable, as the road network is considerably dense and we will only be querying for pivots inside our area of interest.