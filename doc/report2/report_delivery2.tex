\documentclass{report}[a4paper]
\usepackage[top=35mm,bottom=35mm,left=25mm,right=25mm]{geometry} % Margins

% Decent underlines
\usepackage[normalem]{ulem}

% Hyperreferences
\usepackage{hyperref}

% Imports
\usepackage{import}

% Graphics and images
\usepackage{graphicx} \graphicspath{{./images/}}
\usepackage{subcaption}
\usepackage{float}

% Encodings (to render letters with diacritics and special characters)
\usepackage[utf8]{inputenc}

% Language
\usepackage[english]{babel}

% Source code and algorithms
%\usepackage{amsmath}
\usepackage{algorithm}
\usepackage[noend]{algpseudocode}

% Tables with bold rows
\usepackage{tabularx}
\newcommand\setrow[1]{\gdef\rowmac{#1}#1\ignorespaces}
\newcommand\clearrow{\global\let\rowmac\relax}
\clearrow
\usepackage{multirow}

% Lists and items
\usepackage{enumitem}

% Math stuff
\usepackage[mathscr]{euscript}
\usepackage{amssymb, latexsym} %Load math symbols like \blacksquare, but also load normal \leadsto arrows
\usepackage{mathtools} % For \text{...}
% \usepackage{enumitem}
% \usepackage{xcolor}
\newcommand{\expnumber}[2]{{#1}\mathrm{e}{#2}} % scientific notation
\usepackage{siunitx} %SI units
\usepackage{gensymb} % degree symbol

% Definitions, theorems, remarks,...
\usepackage{amsthm}
\newtheorem{definition}{Definition}[section]
\newtheorem{theorem}{Theorem}[section]
\newtheorem{corollary}{Corollary}[theorem]
\newtheorem{lemma}[theorem]{Lemma}
\renewcommand\qedsymbol{$\blacksquare$}
\theoremstyle{remark}
\newtheorem*{remark}{Remark}

% Glossaries
\usepackage[acronym,nonumberlist,entrycounter]{glossaries}
\makeglossaries

\newglossaryentry{heuristic}{
    name=heuristic,
    description={An \emph{heuristic} function $\hat{h}$ of a function $h$ is a function that aims at estimating the value of $h$. In Computer Science, an \emph{heuristic} is a technique to solve a hard problem in a faster way, or to otherwise solve a problem for which there is no known algorithm that can find an exact solution}
}

\newglossaryentry{metaheuristic}{
    name=metaheuristic,
    description={A \emph{metaheuristic} is a strategy to solve certain classes of problems. It is usually a general method to solve a black-box problem, distinguishing it from an \gls{heuristic} which is usually a technique specifically designed to solve a particular instance of a problem}
}

\newglossaryentry{natural selection}{
    name=natural selection,
    description={Key mechanism of biological evolution, based on the observation that individuals' characteristics cause differential survival and reproduction of individuals of a species not subjected to intentional environmental stresses}
}

\newglossaryentry{admissible}{
    name=admissible,
    description={An algorithm is \emph{admissible} iff it always returns an optimal solution, if one exists. A function $\hat{h}$ that estimates $h$ is \emph{admissible} iff it always returns a lower bound for $h$ ($\forall x, \hat{h}(x) < h(x)$)}
}

\newglossaryentry{sweep line}{
    name=sweep line,
    description={A \emph{sweep line} algorithm is a generic approach that represents the search space as a two-dimensional space being swiped by an imaginary line. This is usually implemented by first ordering points by their $x$ coordinate and then iterating over them in that order. The most important observation is that, because geometric operations are limited by metrics, all operations needed by the algorithm only require to analyse the problem near the sweep line.}
}

\newglossaryentry{constructive}{
    name=constructive,
    description={An algorithm is \emph{constructive} when it progressively constructs a solution}
}

\newglossaryentry{iterative}{
    name=iterative,
    description={An heuristic algorithm is \emph{iterative} when it progressively evaluates and improves a complete solution. An algorithm is \emph{iterative} when it repeatedly executes a set of instructions without calling itself.}
}

\newglossaryentry{recursive}{
    name=iterative,
    description={An algorithm is \emph{recursive} when it repeatedly executes a set of instructions by calling itself, as opposed to an \gls{iterative} algorithm.}
}

\newglossaryentry{dynamic programming}{
    name=dynamic programming,
    description={Problem-solving strategy generally used to solve algorithmic problems that have optimal substructure (optimal solutions to subproblems lead to globally optimal solutions) and overlapping subproblems (several subproblems are computed from the same sub-subproblems). The distinguishing feature of this approach is memoization (memorization of subproblem solutions for reuse later on. Generally, wherever there is a recurrence, dynamic programming can be applied}
}

\newglossaryentry{greedy}{
    name=greedy,
    description={An algorithm is \emph{greedy} iff it makes a locally optimal choice in the hope that this choice will lead to a globally optimal solution \cite{intro-alg}. Depending on the problem, they may or may not in fact provide a globally optimal solution.}
}

\newglossaryentry{superpolynomial}{
    name=superpolynomial,
    description={A function $f(n)$ is \emph{superpolynomial} iff $\forall k \in \mathbb{R}, f \not \in O(n^k)$ (i.e., it grows faster than polynomial). This concept is particularly useful when $f$ is superpolynomial but not exponential, like $O(2^{\sqrt{n}})$}
}

\newacronym{DFS}{DFS}{depth-first search}

\newacronym{TSP}{TSP}{travelling salesman problem}

\newacronym{VRP}{VRP}{vehicle routing problem}

\newacronym{SCC}{SCC}{strongly connected component}

\newacronym{TM}{TM}{Turing machine}

\newacronym{NN}{NN}{nearest neighbour}

\renewcommand{\glsentrycounterlabel}{}

% Contents title
\addto\captionsenglish{\renewcommand*\contentsname{Table of contents}}

% Headers and footers
\usepackage{fancyhdr}
\pagestyle{fancyplain}
\fancyhf{}
\lhead{\fancyplain{}{PortoCityTransfers — Delivery II (CAL 2019/20)}}
\rhead{\fancyplain{}{Class 6, group 5}}
\lfoot{\fancyplain{}{\leftmark}}
\rfoot{\thepage}

% Full reference (number and name of section)
\newcommand*{\fullref}[1]{\hyperref[{#1}]{\ref*{#1} \nameref*{#1}}} % One single link

% Email
\newcommand{\email}[1]{
{\texttt{\href{mailto:#1}{#1}} }
}

% Metadata
\title{\Huge PortoCityTransfers \\ \Large Delivery II \\ \vspace*{4pt} \large CAL 2019/20}
\author{
Class 6, group 5 \vspace{0.5em} \\
\begin{tabular}{r l}
	\email{up201806429@fe.up.pt} & Diogo Miguel Ferreira Rodrigues        \\
	\email{up201806613@fe.up.pt} & João António Cardoso Vieira e Basto de Sousa \\
	\email{up201806330@fe.up.pt} & Rafael Soares Ribeiro \\
\end{tabular}
}
\date{22nd of May, 2020}

% Document
\begin{document}
\maketitle
\setcounter{tocdepth}{2}
\tableofcontents
\listoffigures
\listofalgorithms
\clearpage
\printglossary[type=\acronymtype]
\printglossary
\chapter{Introduction} \label{introduction}
This final report details the project designed and implemented in the context of Curricular Unit CAL (Algorithm Design and Analysis).
It fundamentally consists of an application for managing a shuttle service based on a single train station, serving clients in the area inside and surrounding a big city.
\section{Theme description}
The company \emph{PortoCityTransfers} provides shuttle services between Campanhã railway station (Porto, Portugal) and hotels/other places in the Porto region (municipalities of Porto, Matosinhos, Maia, Vila Nova de Gaia). For that end, it owns a certain number of vans, which can transport a given number of passengers each. Service reservation must be made via Internet, where clients can indicate their estimated time of arrival and final destination. Given a set of service requests to satisfy in a given day, and their respective destinations, the company must group passengers and plan routes for that day, having in mind that the passengers should have to wait for a period of time that needs to be as short as possible.\par
This project will design and implement a computer program that will allow \emph{PortoCityTransfers} to plan its routes. The following iterations should be considered for project development:
\begin{enumerate}
    \item The company owns a single van.
    \item The company owns several vans, where multiple travels are simultaneously possible for different vans.
    \item Passengers without reservation must be accounted for, as well as passengers that wish to travel from their homes to the train station.
\end{enumerate}

\chapter{Theoretical notions}
This short chapter presents some basic notions and definitions on Computer Science, which ought to be useful in the following chapters.
\section{Complexity analysis}
Complexity analysis is the process of identifying the computational complexity of an algorithm, describing in a more or less precise way the asymptotic growth of the amount of resources (e.g., time and memory) an algorithm requires.
\begin{definition}[Big-$O$ notation] $f(n) \in O(g(n))$ means that $g$ is an upper bound of $f$.
    \begin{equation*}
        O(g(n))=\{f(n) : \exists c \in \mathbb{R}^+ \colon \exists n_0 \in \mathbb{N} \colon \forall n \geq n_0, f(n) \leq c\cdot g(n)\}
    \end{equation*}
\end{definition}
\begin{definition}[Big-$\Omega$ notation] $f(n) \in \Omega(g(n))$ means that $g$ is a lower bound of $f$.
    \begin{equation*}
        \Omega(g(n))=\{f(n) : \exists c \in \mathbb{R}^+ \colon \exists n_0 \in \mathbb{N} \colon \forall n \geq n_0, f(n) \geq c\cdot g(n)\}
    \end{equation*}
\end{definition}
\begin{definition}[Big-$\Theta$ notation] $f(n) \in \Theta(g(n))$ means that $g$ is a strict upper and lower bound of $f$.
    \begin{equation*}
        \Theta(g(n))=O(g(n)) \cap \Omega(g(n))
    \end{equation*}
\end{definition}
We will not put excessive emphasis on exagerated formalism, so we will by default use the big-$O$ notation.
\section{Graphs}
\subsection{Definitions}
\begin{definition}[Directed weighted graph]
    A directed weighted graph $G$ is a triple $(V, E, w)$, where:
    \begin{itemize}
        \item $V$ is the finite set of \textbf{nodes}.
        \item $E \subseteq V^2$ is the set of \textbf{edges}, where each edge is a pair $(u,v)$ describing an origin and destination.
        \item $w: E \rightarrow \mathbb{R}^+$ is the \textbf{cost function} that maps each edge to a traversal cost.
    \end{itemize}
\end{definition}
\begin{definition}[Transposed graph]
    The transposed graph $G^T(V, E^T)$ of a graph $G(V, E)$ is similar to $G$ but with reversed edge directions:
    \begin{equation*}
        (u, v) \in E \iff (v, u) \in E^T
    \end{equation*}
\end{definition}
\begin{definition}[Path]
    A path $p$ of length $k$ in a graph $G(V,E)$ is a sequence of nodes $\langle p_0,p_1,p_2,...,p_k\rangle$ such that
    \begin{alignat*}{5}
        \text{(All nodes belong to the graph)}       ~~&\forall~0 \leq &&i   &&<    k,&&~p_i \in V \\
        \text{(Node is reachable from previous node)}~~&\forall~0 \leq &&i   &&<    k,&&~(p_i, p_{i+1}) \in E \\
        \text{(No repeated nodes)}                   ~~&\forall~0 \leq &&i,j &&\leq k,&&~i\neq j \implies p_i \neq p_j \\
        \text{(No repeated edges)}                   ~~&\forall~0 \leq &&i,j &&<    k,&&~i\neq j \implies (p_i,p_{i+1}) \neq (p_j, p_{j+1})
    \end{alignat*}
\end{definition}
\begin{definition}[Set of all paths]
    $P(G)$ is the set of all paths in graph $G(V,E)$.
\end{definition}
\begin{definition}[Weight of a path]
    The weight of a path $p$ of length $k$ in graph $G(V,E,w)$ is
    \begin{equation*}
        W(p) = \sum_{i=0}^{k-1}{w(p_i, p_{i+1})}
    \end{equation*}
\end{definition}
\begin{definition}[Adjacency set]
    The adjacency set of $u \in V$ in $G(V,E)$ is the set of nodes directly reachable from $u$:
    \begin{equation*}
        Adj(G, u) = \{(u, v) \in E\} = E \cap (\{v\}\times V)
    \end{equation*}
\end{definition}
\begin{definition}[Descendants] The descendants of a node $s \in V$ in a graph $G(V,E)$ are
    \begin{equation*}
        Desc(s) = Adj(G, s)
    \end{equation*}
\end{definition}
\begin{definition}[Predecessors] The predecessors of a node $s \in V$ in a graph $G(V,E)$ are
    \begin{equation*}
        Pred(s) = Adj(G^T, s)
    \end{equation*}
\end{definition}
\begin{definition}[Strongly connected graph]
    A strongly connected graph is one where there is a path between any two nodes $u, v \in V$.
\end{definition}
\begin{definition}[Strongly connected component]
    A strongly connected component $SCC$ is a subgraph of $G(V, E)$ which is a maximum strongly connected graph (there is no other node $x \not \in SSC$ that can reach and be reached by all nodes $u \in SSC$).
\end{definition}
\begin{definition}[Bridge]
    A bridge in a strongly connected graph $G(V,E)$ is an edge $e \in E$ that, if removed, would disconnect $G$.
\end{definition}
% \section{Miscellaneous}
% \begin{theorem}[Master theorem of Divide-and-Conquer] \label{theor:master}
%     Let $a \geq 1$, $b > 1$ be constants, let $f(n)$ be a function, and let $T(n)$ be defined on the nonnegative integers by the recurrence
%     \begin{equation*}
%         T(n)=a\cdot T(n/b) + f(n)
%     \end{equation*}
%     Then $T(n)$ has the following assymptotic bounds:
%     \begin{enumerate}
%         \item If $f(n) \in O(n^{\log_b a-\epsilon})$ for some constant $\epsilon > 0$, then $T(n) \in \Theta(n^{\log_b a})$
%         \item If $f(n) \in \Theta(n^{\log_b a})$, then $T(n) \in \Theta(f(n) \log n)$
%         \item If $f(n) \in \Omega(n^{\log_b a+\epsilon})$ for some constant $\epsilon > 0$, and $a f(n/b)\leq c f(n)$ for some constant $c<1$ and large $n$, then $T(n) \in \Theta(f(n))$
%     \end{enumerate}
% \end{theorem}
% The proof is prompty available in \cite[p.~97]{intro-alg}.

\chapter{Data structures} \label{data-structures}

\section{Stack}
A stack (\texttt{std::stack}) is a sequencial data structure, where the most recently inserted element is the first to be removed (much like a stack of books, which we only interact with through the top of the stack).\par
We consider a stack $L$ has its top on the left side, and elements \emph{before} $u$ are closer to the top of the stack (or to the left of $u$). Thus, let $Before(v)$ be the set of nodes that are before $v$ in $L$, let $After(v)$ be the set of nodes that are after $v$ in $L$.

\section{Unordered set}
An unordered set (\texttt{std::unordered\_set}) is an associative type of container from the C++ Standard Library (STL). It stores a set of unique objects which are organized into buckets according to their hash values and are not sorted in any particular order. It is implemented as an hash table, allowing amortized constant-time complexity in search, insertion and removal of elements.

\section{Unordered map}
The unordered map data structure (\texttt{std::unordered\_map}) is also an associative container from the STL, used to store key-value pairs (with unique keys). It is implemented as an hash table (where the hash is calculated from the key), providing amortized constant-time complexity for search, insertion and removal operations.

\section{List}
A list is a sequential data structure which links an element to the element coming after it. The STL implementation (\texttt{std::list}) is a doubly-linked list, meaning each element keeps a link to the previous and following elements, so it allows forward- and backward-iteration, as well as constant-time insertion and deletion operations anywhere in the list, meaning it can be easily expanded and reduced.

\section{Vector}
A vector (\texttt{std::vector}) is a sequential data structure that uses a dynamically allocated array to store its elements. Vectors can allocate extra memory for future expansion, consuming more memory than C-arrays but having a much more simple interface. Since elements are sequentially stored in memory, they can be randomly accessed in constant time. Vectors can be reduced in constant time, and increased in amortized constant time, although it is mostly used in situations where the data structure is created once and not increased/reduced very often.

\section{Queue}
A queue (\texttt{std::queue}) is a FIFO data structure, where the first element to be added is processed first than all other elements.

\section{Shared queue}
A shared queue (\texttt{shared\_queue}) is a mere wrapper around a queue, which has a mutex to guarantee certain operations are performed atomically from threads' perspective by guaranteeing only a single thread can perform a certain action at each instant.

\section{Priority queue}
A priority queue (\texttt{std::priority\_queue}) is a data structure allowing insertion of elements associated with a \emph{priority}, and removing the element with the highest priority (if it is a maximum-priority queue). Its elements are non-increasingly ordered, meaning the first element is always the greatest. Therefore, it provides constant-time access to the greatest element, and logarithmic-time insertion and extraction.\par
A minimum-priority queue is similar to a maximum-priority queue, except elements are non-decreasingly ordered, so the first element is always the smallest.

\section{Graph}
We have already presented a theoretical definition of a graph, so now we will only mention our implementations, \texttt{DUGraph} and \texttt{DWGraph}.\par
Both structures represent a graph as an adjacency list, where we map each node to a set/map containing all adjacent nodes/edges. The only difference between \texttt{DUGraph} and \texttt{DWGraph} is that \texttt{DUGraph} stores the set of adjacent nodes for each node, while \texttt{DWGraph} stores for each node $u$ a map of pairs (node $v$, weight $c$), meaning there is an edge from $u$ to $v$ and the cost of going from $u$ to $v$ is $w(u,v)=c$.\par
We additionally keep some other structures, such as a list of all nodes so we can return a constant reference to the list of all nodes in constant time, as well as a mapping from each node to their precedessors, so we can more rapidly delete edges to any node $v$.

\section{Vertical stripes}
The \hyperref[algorithm-vstripes]{vertical stripes algorithm} is more of a data structure than an algorithm. Please check section \ref{algorithm-vstripes} for more details on this data structure/algorithm.
../report1/problem-formalization.tex
../report1/problem-decomposition.tex
\import{./algorithmic-problems/}{algorithmic-problems.tex}
\chapter{Use cases} \label{use-cases}
Our program should allow a user to:
\begin{itemize}
    \item Visualize the road network, placing nodes in their correct geographical position and displaying speed limits in a graphical way, using the suggested tool \href{https://github.com/STEMS-group/GraphViewer}{GraphViewer} or another visualization tool if necessary.
    \item Visualize the \acrshort{SCC} of the train station.
    \item Find the shortest path from one node to another.
    \item Given a list of vans and services for a day, find the set of routes for that day.
\end{itemize}

\chapter{Implemented use cases} \label{implemented-use-cases}
\section{Graphical uses}
All graphical uses make use of \href{https://github.com/STEMS-group/GraphViewer}{GraphViewer}, and are of the form \texttt{./main \textit{cmd} FRACTION FLAGS}, where:
\begin{itemize}
    \item \texttt{FRACTION} is the fraction of roads to draw; more specifically, if \texttt{FRACTION} is $k$, for a given way $w$ the nodes $w_0, w_k, w_{2k},...,w_{|w|}$ are drawn. Thus, the higher the number, the greater the performance is, but also less nodes/edges are drawn.
    \item \texttt{FLAGS} flags which roads are to be drawn (add them to combine):
    \begin{center}
        \begin{tabular}{l | r}
            \textbf{Type of road} & \textbf{Flag} \\ \hline
            Motorway     &    1 \\
            Trunk        &    2 \\
            Primary      &    4 \\
            Secondary    &    8 \\
            Tertiary     &   16 \\
            Road         &   32 \\
            Residential  &   64 \\
            Slow         &  128
        \end{tabular}
    \end{center}
\end{itemize}
\subsection{View}
\texttt{./main view FRACTION FLAGS} draws the road network, colouring roads according to their designated role (highway, residential, ...). Roads are drawn according to the following scheme:
\begin{center}
    \begin{tabular}{l | l}
        \textbf{Type of road} & \textbf{Colour} \\ \hline
        Motorway              & Red          \\
        Trunk                 & Pink         \\
        Primary               & Orange       \\
        Secondary             & Yellow       \\
        Tertiary              & Gray         \\
        Road                  & Gray         \\
        Residential           & Gray         \\
        Slow                  & Gray, dashed
    \end{tabular}
\end{center}

\write18{ if [ ! -f Y8ZJyr3.png ] ; then curl https://i.imgur.com/Y8ZJyr3.png -o Y8ZJyr3.png ; fi ; }
\begin{figure}[H]
    \centering
    \includegraphics[scale=0.213]{Y8ZJyr3}
    \caption{Map with types of roads}
\end{figure}

\subsection{Speed}
\texttt{./main speed FRACTION FLAGS} draws the road network, colouring roads according to their maximum allowed speed. Roads are drawn according to the following scheme:
\begin{center}
    \begin{tabular}{r | l l l l l l}
        \textbf{Speed [$\SI{}{km/h}$] (up to)} & 120 & 100    & 80     & 60    & 50    & 40 \\ \hline
        \textbf{Color}                         & Red & Orange & Yellow & Green & Black & Gray      
        \end{tabular}
\end{center}

\write18{ if [ ! -f Njimj9n.png ] ; then curl https://i.imgur.com/Njimj9n.png -o Njimj9n.png ; fi ; }
\begin{figure}[H]
    \centering
    \includegraphics[scale=0.213]{Njimj9n}
    \caption{Map with types of roads}
\end{figure}

\subsection{Strongly connected components}
\texttt{./main scc FRACTION FLAGS} draws the road network, colouring roads red if they connect two nodes in the train station's \acrshort{SCC}, or gray if at least one of the nodes is not in the train station's \acrshort{SCC}.
\subsection{Shortest path}
\texttt{./main path FRACTION FLAGS SOUR DEST [-v]} draws the road network, colouring the shortest path (actually least-time path) from nodes \texttt{SOUR} to \texttt{DEST} found by different algorithms. Using the option \texttt{-v} ignores all paths other than the optimal path, and colours nodes if they were explored by certain algorithms.\par

\texttt{./main scc FRACTION FLAGS SOUR\_LAT SOUR\_LON DEST\_LAT DEST\_LON [-v]} is similar to the above usage, except you can provide latitude and longitude coordinates for the source and destination, and an extra step will be performed to find the corresponding source and destination nodes that are 1. closest to the provided coordinates, and 2. connected to the train station.\par

The A* versions are distinguished by the maximum speed they assume a car can go. This is just a method to guarantee we have admissible (only A* algorithm $\SI{90}{km/h}$) and non-admissible heuristics.
\begin{center}
    \begin{tabular}{l | l | l}
        \textbf{Algorithm}            & \textbf{Path colour} & \textbf{Visited nodes colour} \\ \hline
        Dijkstra's algorithm with \hyperref[alg:dijkstra-early-stop]{early stop} & Black                & Pink                          \\
        A* algorithm, $\SI{90}{km/h}$ & Black                & Red                           \\
        A* algorithm, $\SI{60}{km/h}$ & Magenta              & Magenta                       \\
        A* algorithm, $\SI{30}{km/h}$ & Blue                 & Blue                          \\
        A* algorithm, $\SI{10}{km/h}$ & Cyan                 & Cyan                          
    \end{tabular}
\end{center}

\write18{ if [ ! -f OdwmfwK.png ] ; then curl https://i.imgur.com/OdwmfwK.png -o OdwmfwK.png ; fi ; }
\write18{ if [ ! -f YTZbVeR.png ] ; then curl https://i.imgur.com/YTZbVeR.png -o YTZbVeR.png ; fi ; }
\begin{figure}[h]
    \centering
    \begin{subfigure}{.49\textwidth}
        \centering
        \includegraphics[trim={420px 0 600px 0}, clip, scale=0.255]{OdwmfwK}
        \caption{Without \texttt{-v}}
    \end{subfigure}
    \begin{subfigure}{.49\textwidth}
        \centering
        \includegraphics[trim={420px 0 600px 0}, clip, scale=0.255]{YTZbVeR}
        \caption{With \texttt{-v}}
    \end{subfigure}
    \caption{Map with shortest path from \textit{Senhor da Pedra} beach (V. N. Gaia) to Porto Airport (Maia).}
\end{figure}

If \texttt{-v} is not used and a road belongs to more than one of the paths found by the different algorithms, that road is coloured with the colour of the best path (e.g., a road belongs to the Blue and Magenta paths, then it is coloured magenta).\par
If \texttt{-v} is used and a road is explored by more than one of the algorithms, that road is coloured with the colour of the worst path (e.g., a road is explored by the Blue and Magenta algorithms, then it is coloured blue).\par
The colouring scheme for \texttt{-v} is based on the assumption that algorithms providing better answers explore more nodes, so to visualize all sets of explored edges we may assume in most cases that the set of edges explored by a worse algorithm are a subset of edges explored by a better algorithm, and then the user may assume, for instance, that all nodes painted in Blue were explored by the Magenta, Red and Pink algorithms.

\section{Iterations}
\texttt{./main (iteration1 | iteration2) CLIENTS VANS RIDES} loads clients and vans from files \texttt{CLIENTS} and \texttt{RIDES}, processes them to allocate clients and vans to rides, and prints the resulting rides to file \texttt{RIDES}. 

\section{To implement}
\begin{itemize}
    \item Given a rides file and a ride number, st of vans and services for a day, find the set of routes for that day.
\end{itemize}

\chapter{Graph connectivity} \label{connectivity-graphs}
During a first visual inspection of the graphs it was obvious that not every road  would be able to reach the train station, being the pre-processing of the graphs a necessity.
For this effect, we used the \hyperref[algorithm-scc-kosaraju-v]{Variant of the Kosaraju algorithm} that was mentioned in the Algorithmic Problems chapter, which we thought to be the most reliable option.
Using the method getConnectedGraph (MapGraph.cpp) that applies to the original graph this processing, we obtain a graph that is made up of 209872 nodes, which is a decrease close to 2 per cent of the number of nodes when comparing to the  213367 that the graph returned by getFullGraph (MapGraph.cpp) has.
\chapter{Iterations' descriptions} \label{iterations}

All iterations work in four fundamentally independent stages:
\begin{enumerate}
    \item Get the \textbf{connected graph}.
    \item For each client coordinate, find the \textbf{closest node}.
    \item Find the \textbf{shortest paths} betwee all pairs of interesting nodes (train station and all nodes where clients need to be picked up).
    \item Solve an instance of the \textbf{\acrlong*{TSP}}.
\end{enumerate}

In the end, results are printed to a file, which should be named with the extension \texttt{.rides}.\par

Let $C$ be the number of clients.

\section{Iteration I}
This iteration starts by processing the ``full" graph in order to get its ``connected version". This pre-processing is achieved by applying the \hyperref[algorithm-scc-kosaraju-v]{variant of Kosaraju's algorithm} which has a time complexity of $O(|E| + |V|)$.\par
We then find for each client the closest node in the connected graph, using the \hyperref[algorithm-vstripes]{vertical stripes algorithm}, which is initialized once in time $O(\Delta x / \Delta + |V| \log |V|)$. In our case, we used $\Delta = 0.025\degree$, and since $\Delta x = 0.25\degree$ the first term is constant and neglibible, thus the time complexity is actually $O(|V| \log |V|)$. To perform the $C$ queries (one for each client), one needs time $O(C \log |V|)$.\par
Afterwards, several instances of \hyperref[algorithm-shortestpath-dijkstra]{Dijkstra's algorithm} are run in paralell to get the length of the shortest paths between all clients' destinations and the station


O) that takes advantage of multithreading (more precisely 8 threads). This algorithm runs in $O(|E|\cdot \text{T}_{dk}+|V|\cdot \text{T}_{em})$ and due to the high number of calculations that are performed the multithreading implementation may not have significant impact in performance (NEED HELP HERE)
\par
To finish, it resorts to the Held Karp algorithm that detains a $O(|V|^2 \cdot 2^{|V|})$ complexity to find the optimal route given the clients, the van, the station and the shortest paths between nodes previously calculated, determining for the iteration a time complexity in the order of $O(|V|^2 \cdot 2^{|V|})$.

\section{Iteration II}
The second iteration 

\section{Iteration III}
The third iteration was not implemented due to its complexity.


\chapter{Conclusion} \label{conclusion}
After briefly presenting the problem statement, we formalized and decomposed the problem into more fundamental problems, which range from simple problems such as reachability, shortest path-finding and \acrshortpl{SCC}, to more complex ones such as the \acrshort{TSP} and \acrshort{VRP}. For each algorithm, we performed complexity analysis and compared results to theoretically evaluate their relative performances and, after considering the pros and cons of each approach, point to a particular solution that will most likely solve our problem in the generally best way.\par
Our fundamental approach is that, once we decompose the main problem into fundamental, known problems, we can then focus on atomic solutions which can be easily tested and benchmarked. We start by solving the shortest path problem, essentially because this is arguably the most important problem in maps: how do I get from a certain place to another in the least time possible?\par
We then go on to build an increasingly complete algorithms database that we will most likely use in the second part of the project, building on atomic solutions to reach successively more complex solutions (for instance, after knowing how to find the shortest path, we need to organize travels so that the system is globally optimized, or at least to a certain extent).\par
During the process of writting this report, we have used numerous concepts familiar to Computer Science and Algorithmics and presented in the context of CAL theoretical classes or researched in the context of this project, such as \emph{\hyperref[glsentry-recursive]{recursion}}, \emph{\hyperref[glsentry-iterative]{iteration}}, \emph{\hyperref[teor:dfs]{mathematical induction}}, \emph{\hyperref[sec:dijkstra-PoC]{cycle variants/invariants}}, \emph{\gls{dynamic programming}}, \emph{\hyperref[glsentry-heuristic]{heuristics}}, \emph{\hyperref[glsentry-greedy]{greedy solutions}}, \emph{\hyperref[algorithm-reachability-dfs]{backtracking}} and \emph{\hyperref[complexity-analysis]{time/memory complexity}}.\par

In the next project delivery we aim at implementing and testing these algorithms, as well as using them in the actual problem we are facing.\par
!REVIEW.
\section{Tasks allocation}
\begin{itemize}
    \item 'D': done
    \item 'R': review
    \item 'D/R': almost done, missing final review
    \item ' ': undone
\end{itemize}
\begin{center}
    \begin{tabular}{l | c | p{29mm} p{30mm} p{29mm}}
        Sections                                        &       & Diogo Rodrigues & João António Sousa & Rafael Ribeiro \\ \hline
        \fullref{introduction}                          & D     & All & -   & -   \\
        \fullref{theoretical-notions}                   & D     & Most & -   & Reachability \\
        \fullref{problem-formalization}                 & D     & All & -   & -   \\
        \fullref{problem-decomposition}                 & D/R   & All & -   & -   \\
        \fullref{algorithm-reachability-dfs}            & D     & All & -   & -   \\
        \fullref{algorithm-shortestpath-floydwarshall}  & D     & -   & All & -   \\
        \fullref{algorithm-shortestpath-dijkstra}       & D     & All & -   & -   \\
        \fullref{algorithm-shortestpath-astar}          & D     & All & -   & -   \\
        \fullref{algorithm-scc-kosaraju}                & D     & Proof of correctness, variant & -   & Most \\
        \fullref{algorithm-scc-tarjan}                  & D     & Proof of correctness & -   & Most \\
        \fullref{algorithm-tsp-heldkarp}                & D     & All & -   & -   \\
        \fullref{algorithm-tsp-nn}                      & D     & -   & All & -   \\
        \fullref{algorithm-vrp-optimal}                 & D     & -   & All & -   \\
        \fullref{algorithm-vrp-heuristic}               & D     & All & -   & -   \\
        \fullref{algorithm-vrp-advanced}                & D     & All & -   & -   \\
        \fullref{use-cases}                             & D/R   & -   & -   & -   \\
        \fullref{conclusion}                            & D/R   & -   & -   & -   \\
    \end{tabular}
\end{center}


\bibliographystyle{acm}
\addcontentsline{toc}{chapter}{Bibliography}
\bibliography{report}
\end{document}
