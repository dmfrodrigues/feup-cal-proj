\chapter{Implemented use cases} \label{implemented-use-cases}
\section{Graphical uses}
All graphical uses make use of \href{https://github.com/STEMS-group/GraphViewer}{GraphViewer}, and are of the form \texttt{./main \textit{cmd} FRACTION FLAGS}, where:
\begin{itemize}
    \item \texttt{FRACTION} is the fraction of roads to draw; more specifically, if \texttt{FRACTION} is $k$, for a given way $w$ the nodes $w_0, w_k, w_{2k},...,w_{|w|}$ are drawn. Thus, the higher the number, the greater the performance is, but also less nodes/edges are drawn.
    \item \texttt{FLAGS} flags which roads are to be drawn (add them to combine):
    \begin{center}
        \begin{tabular}{l | r}
            \textbf{Type of road} & \textbf{Flag} \\ \hline
            Motorway     &    1 \\
            Trunk        &    2 \\
            Primary      &    4 \\
            Secondary    &    8 \\
            Tertiary     &   16 \\
            Road         &   32 \\
            Residential  &   64 \\
            Slow         &  128
        \end{tabular}
    \end{center}
\end{itemize}
\subsection{View}
\texttt{./main view FRACTION FLAGS} draws the road network, colouring roads according to their designated role (highway, residential, ...). Roads are drawn according to the following scheme:
\begin{center}
    \begin{tabular}{l | l}
        \textbf{Type of road} & \textbf{Colour} \\ \hline
        Motorway              & Red          \\
        Trunk                 & Pink         \\
        Primary               & Orange       \\
        Secondary             & Yellow       \\
        Tertiary              & Gray         \\
        Road                  & Gray         \\
        Residential           & Gray         \\
        Slow                  & Gray, dashed
    \end{tabular}
\end{center}

\write18{ if [ ! -f Y8ZJyr3.png ] ; then curl https://i.imgur.com/Y8ZJyr3.png -o Y8ZJyr3.png ; fi ; }
\begin{figure}[H]
    \centering
    \includegraphics[scale=0.213]{Y8ZJyr3}
    \caption{Map with types of roads}
\end{figure}

\subsection{Speed}
\texttt{./main speed FRACTION FLAGS} draws the road network, colouring roads according to their maximum allowed speed. Roads are drawn according to the following scheme:
\begin{center}
    \begin{tabular}{r | l l l l l l}
        \textbf{Speed [$\SI{}{km/h}$] (up to)} & 120 & 100    & 80     & 60    & 50    & 40 \\ \hline
        \textbf{Color}                         & Red & Orange & Yellow & Green & Black & Gray      
        \end{tabular}
\end{center}

\write18{ if [ ! -f Njimj9n.png ] ; then curl https://i.imgur.com/Njimj9n.png -o Njimj9n.png ; fi ; }
\begin{figure}[H]
    \centering
    \includegraphics[scale=0.213]{Njimj9n}
    \caption{Map with road speeds}
\end{figure}

\subsection{Strongly connected components}
\texttt{./main scc FRACTION FLAGS} draws the road network, colouring roads red if they connect two nodes in the train station's \acrshort{SCC}, or gray if at least one of the nodes is not in the train station's \acrshort{SCC}.
\subsection{Shortest path}
\texttt{./main path FRACTION FLAGS SOUR DEST [-v]} draws the road network, colouring the shortest path (actually least-time path) from nodes \texttt{SOUR} to \texttt{DEST} found by different algorithms. Using the option \texttt{-v} ignores all paths other than the optimal path, and colours nodes if they were explored by certain algorithms.\par

\texttt{./main scc FRACTION FLAGS SOUR\_LAT SOUR\_LON DEST\_LAT DEST\_LON [-v]} is similar to the above usage, except you can provide latitude and longitude coordinates for the source and destination, and an extra step will be performed to find the corresponding source and destination nodes that are 1. closest to the provided coordinates, and 2. connected to the train station.\par

The A* versions are distinguished by the maximum speed they assume a car can go. This is just a method to guarantee we have admissible (only Dijkstra and A* algorithm $\SI{90}{km/h}$) and non-admissible heuristics.
\begin{center}
    \begin{tabular}{l | l | l}
        \textbf{Algorithm}            & \textbf{Path colour} & \textbf{Visited nodes colour} \\ \hline
        Dijkstra's algorithm with \hyperref[alg:dijkstra-early-stop]{early stop} & Black                & Pink                          \\
        A* algorithm, $\SI{90}{km/h}$ & Black                & Red                           \\
        A* algorithm, $\SI{60}{km/h}$ & Magenta              & Magenta                       \\
        A* algorithm, $\SI{30}{km/h}$ & Blue                 & Blue                          \\
        A* algorithm, $\SI{10}{km/h}$ & Cyan                 & Cyan                          
    \end{tabular}
\end{center}

\write18{ if [ ! -f OdwmfwK.png ] ; then curl https://i.imgur.com/OdwmfwK.png -o OdwmfwK.png ; fi ; }
\write18{ if [ ! -f YTZbVeR.png ] ; then curl https://i.imgur.com/YTZbVeR.png -o YTZbVeR.png ; fi ; }
\begin{figure}[h]
    \centering
    \begin{subfigure}{.49\textwidth}
        \centering
        \includegraphics[trim={420px 0 600px 0}, clip, scale=0.255]{OdwmfwK}
        \caption{Without \texttt{-v}}
    \end{subfigure}
    \begin{subfigure}{.49\textwidth}
        \centering
        \includegraphics[trim={420px 0 600px 0}, clip, scale=0.255]{YTZbVeR}
        \caption{With \texttt{-v}}
    \end{subfigure}
    \caption{Map with shortest path from \textit{Senhor da Pedra} beach (V. N. Gaia) to Porto Airport (Maia).}
\end{figure}

If \texttt{-v} is not used and a road belongs to more than one of the paths found by the different algorithms, that road is coloured with the colour of the best path (e.g., a road belongs to the Blue and Magenta paths, then it is coloured magenta).\par
If \texttt{-v} is used and a road is explored by more than one of the algorithms, that road is coloured with the colour of the worst path (e.g., a road is explored by the Blue and Magenta algorithms, then it is coloured blue).\par
The colouring scheme for \texttt{-v} is based on the assumption that algorithms providing better answers explore more nodes, so to visualize all sets of explored edges we may assume in most cases that the set of edges explored by a worse algorithm are a subset of edges explored by a better algorithm, and then the user may assume, for instance, that all nodes painted in Blue were explored by the Magenta, Red and Pink algorithms.

\subsection{Reduced graph}
\texttt{./main reduced} allows to see the reduced version of the connected graph. This reduced graph was an early effort with the objective of rapidly finding the distance between any two nodes, by reducing the original graph by removing middle nodes in one-way and two-way roads.\par
We were able to reduce the original graph (with $209774$ nodes and $386249$ edges) to a reduced graph with $42403$ nodes and $97249$ edges through an iterative process, a very significant decrease but nonetheless not enough to make it feasible to calculate the distance between all pairs of nodes in the reduced graph, even when using several threads (this operation would take an estimated $\SI{6}{min}$ to complete using a 12-thread CPU, which we considered to be excessive).

\write18{ if [ ! -f BRxpvq4.png ] ; then curl https://i.imgur.com/BRxpvq4.png -o BRxpvq4.png ; fi ; }
\begin{figure}[H]
    \centering
    \includegraphics[scale=0.24]{BRxpvq4}
    \caption[Map close-up of the reduced graph near Porto city centre]{Map close-up of the reduced graph near Porto city centre. Notice how \emph{Rotunda da Boavista} is drawn with the least amount of nodes possible, and how Freixo Bridge's lanes are not exactly paralell, since the two ways interface with other roads in the north end of the bridge using nodes that are relatively appart from each other.}
\end{figure}

\subsection{Rides}
\texttt{./main ride FRACTION FLAGS RIDES NUM\_RIDES} allows to visualize ride number \texttt{NUM\_RIDES} from rides file \texttt{RIDES}. Roads traversed while performing this ride are coloured red, and the train station and places where clients are dropped are marked with black circles.

\write18{ if [ ! -f yEcn4Gi.png ] ; then curl https://i.imgur.com/yEcn4Gi.png -o yEcn4Gi.png ; fi ; }
\begin{figure}[H]
    \centering
    \includegraphics[trim={230px 0 230px 0}, clip, scale=0.27]{yEcn4Gi}
    \caption[Map of a ride]{Map of a ride, obtained by running \texttt{./main ride 1 255 resources/it1\_03.rides 1}. Check the \texttt{README.md} file to find how to create file \texttt{resources/it1\_03.rides} using the suggested examples.}
\end{figure}

\section{Iterations}
\texttt{./main (iteration1 | iteration2) CLIENTS VANS RIDES} loads clients and vans from files \texttt{CLIENTS} and \texttt{VANS}, processes them to allocate clients and vans to rides, and prints the resulting rides to file \texttt{RIDES}. 
