\chapter{Introduction} \label{introduction}
This report details the project to be designed and implemented in the context of Curricular Unit CAL (Algorithm Design and Analysis).
It fundamentally consists of an application for managing a shuttle service based on a single train station, serving clients in the area inside and surrounding a big city.
\section{Theme description}
The company \emph{PortoCityTransfers} provides shuttle services between Campanhã railway station (Porto, Portugal) and hotels/other places in the Porto region (municipalities of Porto, Matosinhos, Maia, Vila Nova de Gaia). For that end, it owns a certain number of vans, which can transport a given number of passengers each. Service reservation must be made via Internet, where clients can indicate their estimated time of arrival and final destination. Given a set of service requests to satisfy in a given day, and their respective destinations, the company must group passengers and plan routes for that day, having in mind that the passengers should have to wait for a period of time that needs to be as short as possible.\par
This project will design and implement a computer system that will allow \emph{PortoCityTransfers} to plan its routes. The following iterations should be considered for project development:
\begin{enumerate}
    \item The company owns a single van.
    \item The company owns several vans, where multiple travels are simultaneously possible for different vans.
    \item Passengers without reservation must be accounted for, as well as passengers that wish to travel from their homes to the train station.
\end{enumerate}
