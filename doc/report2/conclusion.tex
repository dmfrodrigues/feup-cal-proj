\chapter{Conclusion} \label{conclusion}
After briefly presenting the problem statement, we formalized and decomposed the problem into more fundamental problems, which range from simple problems such as reachability, shortest path-finding and \acrshortpl{SCC}, to more complex ones such as the \acrshort{TSP} and \acrshort{VRP}. For each algorithm, we performed complexity analysis and compared results to theoretically evaluate their relative performances and, after considering the pros and cons of each approach, point to a particular solution that will most likely solve our problem in the generally best way.\par
Our fundamental approach is that, once we decompose the main problem into fundamental, known problems, we can then focus on atomic solutions which can be easily tested and benchmarked. We start by solving the shortest path problem, essentially because this is arguably the most important problem in maps: how do I get from a certain place to another in the least time possible?\par
We then go on to build an increasingly complete algorithms database that we will most likely use in the second part of the project, building on atomic solutions to reach successively more complex solutions (for instance, after knowing how to find the shortest path, we need to organize travels so that the system is globally optimized, or at least to a certain extent).\par
During the process of writting this report, we have used numerous concepts familiar to Computer Science and Algorithmics and presented in the context of CAL theoretical classes or researched in the context of this project, such as \emph{\hyperref[glsentry-recursive]{recursion}}, \emph{\hyperref[glsentry-iterative]{iteration}}, \emph{\hyperref[teor:dfs]{mathematical induction}}, \emph{\hyperref[sec:dijkstra-PoC]{cycle variants/invariants}}, \emph{\gls{dynamic programming}}, \emph{\hyperref[glsentry-heuristic]{heuristics}}, \emph{\hyperref[glsentry-greedy]{greedy solutions}}, \emph{\hyperref[algorithm-reachability-dfs]{backtracking}} and \emph{\hyperref[complexity-analysis]{time/memory complexity}}.\par

We would like to highlight that the implementation of multithreading in the Dijkstra's algorithm, as expected, yielded performance improvements, especially for a number of clients lower or equal to the maximum number of threads being used at each instant (eight). However that efficiency improvement did not translate to a better time complexity of the algorithm due to the fact of it being a constant.
Another point we would like to address relative to the Dijkstra's algorithm is that a Fibonacci Heap implementation while in theory would perform better than a Binary Heap revealed to be close to two times slower than the latter (CAUSE?)

% In the next project delivery we aim at implementing and testing these algorithms, as well as using them in the actual problem we are facing.\par

\par


\section{Tasks allocation}
\subsection{Report}
\begin{center}
    \begin{tabular}{l | p{32mm} p{32mm} p{32mm}}
        Sections                                        & Diogo Rodrigues & João António Sousa & Rafael Ribeiro \\ \hline
        \fullref{introduction}                          & All & -   & -   \\
        \fullref{theoretical-notions}                   & Most & -   & Reachability \\
        \fullref{problem-formalization}                 & All & -   & -   \\
        \fullref{problem-decomposition}                 & Most & Small contribution   & -   \\
        \fullref{algorithm-reachability-dfs}            & All & -   & -   \\
        \fullref{algorithm-shortestpath-floydwarshall}  & -   & All & -   \\
        \fullref{algorithm-shortestpath-dijkstra}       & All & -   & -   \\
        \fullref{algorithm-shortestpath-astar}          & All & -   & -   \\
        \fullref{algorithm-scc-kosaraju}                & Proof of correctness, variant & -   & Most \\
        \fullref{algorithm-scc-tarjan}                  & Proof of correctness & -   & Most \\
        \fullref{algorithm-tsp-heldkarp}                & All & -   & -   \\
        \fullref{algorithm-tsp-nn}                      & -   & All & -   \\
        \fullref{algorithm-vrp-optimal}                 & -   & All & -   \\
        \fullref{algorithm-vrp-heuristic}               & All & -   & -   \\
        \fullref{algorithm-vrp-advanced}                & All & -   & -   \\
        \fullref{use-cases}                             & All & -   & -   \\
        \fullref{conclusion}                            & Most & Small contribution   & -   \\
    \end{tabular}
\end{center}
\subsection{Code}
\begin{center}
    \begin{tabular}{l | p{32mm} p{32mm} p{32mm}}
        Features & Diogo Rodrigues & João António Sousa & Rafael Ribeiro \\ \hline
                 & -               & -                  & -              \\
                 & -               & -                  & -
    \end{tabular}
\end{center}
