\chapter{Graph connectivity} \label{connectivity-graphs}
On visual inspection of the graph, it was obvious that not every node would be reachable from the trains station, making graph pre-processing a necessity to remove unreachable nodes.
For this effect, we used the \hyperref[algorithm-scc-kosaraju-v]{variant of Kosaraju's algorithm}, which we considered to be the most reliable option, both because its complexity was equivalent to other \acrshort{SCC} algorithm alternatives and also since its simple implementation left no doubt about its correct implementation.\par
Using method \texttt{MapGraph::getConnectedGraph} (\texttt{MapGraph.cpp}), we applied this processing to the full graph, obtaining a connected graph with $209872$ nodes (a reduction of $1.64\%$ from $213367$) and $386249$ edges (a reduction of $1.21\%$ from $390983$), a significant improvement from the full graph obtained from \texttt{MapGraph::getFullGraph} (\texttt{MapGraph.cpp}), and a very important stage to avoid unreachability problems.
\write18{ if [ ! -f 3CndwLU.png ] ; then wget https://i.imgur.com/3CndwLU.png ; fi ; }
\write18{ if [ ! -f YwCUPvM.png ] ; then wget https://i.imgur.com/YwCUPvM.png ; fi ; }
\write18{ if [ ! -f faPYR6B.png ] ; then wget https://i.imgur.com/faPYR6B.png ; fi ; }
\begin{figure}[h]
    \centering
    \begin{subfigure}{.50\textwidth}
        \centering
        \includegraphics[trim={460px 0 460px 0}, clip, scale=0.227]{3CndwLU}
        \caption{Complete view}
    \end{subfigure}
    \begin{subfigure}{.49\textwidth}
        \centering
        \includegraphics[scale=0.102]{YwCUPvM}
        \caption{Detail, Freixo bridge in the center}
        \vspace{0.6em}
        \includegraphics[scale=0.102]{faPYR6B}
        \caption{Detail, A41 and A32 (Sandim, V. N. Gaia)}
    \end{subfigure}
    \caption{Connected nodes in map. Although most nodes might look reachable (a), there are certain parts that are not reachable, mostly due to poor mapping or because they travel beyond the map's edges.}
\end{figure}
