\chapter{Use cases} \label{use-cases}
\section{Graphical uses}
All graphical uses make use of \href{https://github.com/STEMS-group/GraphViewer}{GraphViewer}, and are of the form \texttt{./main \textit{cmd} FRACTION FLAGS}, where:
\begin{itemize}
    \item \texttt{FRACTION} is the fraction of roads to draw; more specifically, if \texttt{FRACTION} is $k$, for a given way $w$ the nodes $w_0, w_k, w_{2k},...,w_{|w|}$ are drawn. Thus, the higher the number, the greater the performance is, but also less nodes/edges are drawn.
    \item \texttt{FLAGS} flags which roads are to be drawn (add them to combine):
    \begin{center}
        \begin{tabular}{l | r}
            \textbf{Type of road} & \textbf{Flag} \\ \hline
            Motorway     &    1 \\
            Trunk        &    2 \\
            Primary      &    4 \\
            Secondary    &    8 \\
            Tertiary     &   16 \\
            Road         &   32 \\
            Residential  &   64 \\
            Slow         &  128
        \end{tabular}
    \end{center}
\end{itemize}
\subsection{View}
\texttt{./main view FRACTION FLAGS} draws the road network, colouring roads according to their designated role (highway, residential, ...). Roads are drawn according to the following scheme:
\begin{center}
    \begin{tabular}{l | l}
        \textbf{Type of road} & \textbf{Colour} \\ \hline
        Motorway              & Red          \\
        Trunk                 & Pink         \\
        Primary               & Orange       \\
        Secondary             & Yellow       \\
        Tertiary              & Gray         \\
        Road                  & Gray         \\
        Residential           & Gray         \\
        Slow                  & Gray, dashed
    \end{tabular}
\end{center}
\subsection{Speed}
\texttt{./main speed FRACTION FLAGS} draws the road network, colouring roads according to their maximum allowed speed. Roads are drawn according to the following scheme:
\begin{center}
    \begin{tabular}{r | l}
        \textbf{Speed [$\SI{}{km/h}$] (up to)} & \textbf{Colour} \\ \hline
                                  120 & Red             \\
                                  100 & Orange          \\
                                   80 & Yellow          \\
                                   60 & Green           \\
                                   50 & Black           \\
                                   40 & Gray            
    \end{tabular}
\end{center}
\subsection{Strongly connected components}
\texttt{./main scc FRACTION FLAGS} draws the road network, colouring roads red if they connect two nodes in the train station's \acrshort{SCC}, or gray if at least one of the nodes is not in the train station's \acrshort{SCC}.
\subsection{Shortest path}
\texttt{./main path FRACTION FLAGS SOUR DEST} draws the road network, colouring roads red if they connect two nodes in the shortest path (actually least-time path) from nodes \texttt{SOUR} to \texttt{DEST}.

\section{To implement}
\begin{itemize}
    \item Given a list of vans and services for a day, find the set of routes for that day.
\end{itemize}
